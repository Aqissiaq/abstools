\chapter{Annotations}
ABS supports \emph{Annotations} to enrich an ABS model with additional information, for example, to realize pluggable type systems.
Annotations can appear before any declaration and type usage in ABS programs (which is not given in the grammar definitions, to improve readability).

\begin{abssyntax}
\NT{Annotation} \defn \TRS{[}\ \OPTG{\NT{TypeName}\ \TRS{:}}\ \NT{PureExp}\ \TRS{]}
\end{abssyntax}

\begin{absexample}
[LocationType:Near] Peer p;
[Far] Network n;
List<[Near] Peer> peers = Nil;
\end{absexample}

\section{Type Annotations}
ABS has a predefined meta-annotation \absinline{TypeAnnotation} to declare
annotations to be \emph{Type Annotations}.
Data types that are annotated with that annotation are specially treated by the
ABS compiler to support an easier implementation of pluggable type systems.

\begin{absexample}
[TypeAnnotation]
data LocationType = Far | Near | Somewhere | Infer;
\end{absexample}
