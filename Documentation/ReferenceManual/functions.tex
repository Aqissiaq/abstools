\chapter{Functions}
\emph{Functions} in ABS define names for parametrized data expressions.
A function in ABS is always side effect-free, which means that it cannot
manipulate the heap.

\begin{abssyntax}
\NT{FunctionDecl} \defn \TR{def}\ \TR{Type}\ \NT{Identifier}\ \OPTG{\TRS{<}\ \NT{TypeIdList}\ \TRS{>}}\ \TRS{(} \NT{ParamList} \TRS{)}\ \TRS{=}\ 
                 \NT{FunBody}\ \TRS{;}  \\
\NT{FunBody}      \defn \TR{builtin} ~|~ \NT{PureExp}\\
\NT{TypeIdList}   \defn \NT{TypeId}\ \MANYG{\TRS{,}\ \NT{TypeId}}
\end{abssyntax}

\begin{absexample}
def Int length(IntList list) =
  case list { 
    Nil => 0;
    Cons(n, ls) => 1 + length(ls);
  };
\end{absexample}

\section{Parametric Functions}
\label{sec:parametric-functions}

\emph{Parametric Functions} allow to work with parametric data types in a
general way.  For example, given a list of any type, a parametric
function \absinline{head} can return the first element, regardless of its
type.  Parametric functions are defined like normal functions but have
an additional \texttt{type parameter} section inside angle brackets
(\texttt{< >}) after the function name.

\begin{absexample}
def A head<A>(List<A> list) =
  case list {
    Cons(x, xs) => x;
  };
\end{absexample}

\noindent
(Note that \absinline{head} is a partial function.)