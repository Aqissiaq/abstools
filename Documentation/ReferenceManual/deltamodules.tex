\chapter{Delta Modules}
\label{ch:deltas}

ABS supports the delta-oriented programming model~\cite{SchaeferBBDT10}, an
approach that aids the development of a set of programs simultaneously from a
single code base, following the software product line engineering
approach~\cite{PohlBL05}.
In delta-oriented programming, features defined by a feature model are
associated with code modules that describe modifications to a core program. In
ABS, these modules are called \emph{delta modules}. Hence the implementation of
a software product line in ABS is divided into a \emph{core} and a set of delta
modules.

The core consists of a set of ABS modules containing the classes that implement
a complete software product of the corresponding software product line. Delta
modules (or \emph{deltas} in short) describe how to change the core program to
obtain new products. This includes adding new classes and interfaces, modifying
existing ones, or even removing some classes from the core. Delta modules can
also modify the functional entities of an ABS program, that is, they can add and
modify data types and type synonyms, and add functions.

Deltas are applied to the core program by the ABS compiler front end. The choice
of which delta modules to apply depends on the selection of a set of features,
that is, a particular product of the SPL.
The role of the ABS compiler front end is to translate textual ABS models into
an internal representation and check the models for syntax and semantic errors.
The role of the compiler back end is to generate code for the models targeting
some suitable execution or simulation environment.


\section{Syntax}
\label{delta syntax}
\begin{figure}
    \centering
    \begin{tabular}{rcl}

        \NT{DeltaDecl}
        \concrDefn{ \TR{delta} \NT{TypeId} \OPT{\NT{DeltaParams}} \TRS{;} \OPT{\NT{ModuleAccess}} \MANY{\NT{ModuleModifier}} }
        
        \\
        \NT{ModuleModifier}
        \concrDefn{ \TR{adds} \NT{ClassDecl} }
        \concrCont{ \TR{removes} \TR{class} \NT{TypeName} \TRS{;} }
        \concrCont{ \TR{modifies} \TR{class} \NT{TypeName} }
        \concrNewline{ \OPTG{\TR{adds}\ \NT{TypeId}\ \MANYG{\TRS{,}\ \NT{TypeId}}} \OPTG{\TR{removes}\ \NT{TypeId}\ \MANYG{\TRS{,}\ \NT{TypeId}}} }
        \concrNewline{ \TRS{\{} \MANY{\NT{Modifier}} \TRS{\}} }
        \concrCont{ \TR{adds} \NT{InterfaceDecl} }
        \concrCont{ \TR{removes} \TR{interface} \NT{TypeName} \TRS{;} }
        \concrCont{ \TR{modifies} \TR{interface} \NT{TypeName} \TRS{\{} \MANY{\NT{InterfaceModifier}} \TRS{\}}}
        \concrCont{ \TR{adds} \NT{FunctionDecl} }
        \concrCont{ \TR{adds} \NT{DataTypeDecl} }
        \concrCont{ \TR{modifies} \NT{DataTypeDecl} }
        \concrCont{ \TR{adds} \NT{TypeSynDecl} }
        \concrCont{ \TR{modifies} \NT{TypeSynDecl} }
        \concrCont{ \TR{adds} \NT{Import} }
        \concrCont{ \TR{adds} \NT{Export} }
        \medskip
        \\

        \NT{InterfaceModifier}
        \concrDefn{ \TR{adds} \NT{MethSig} \TRS{;} }
        \concrCont{ \TR{removes} \NT{MethSig} \TRS{;} }
        \medskip
        \\

        \NT{Modifier}
        \concrDefn{ \TR{adds} \NT{FieldDecl}}
        \concrCont{ \TR{removes} \NT{FieldDecl}}
        \concrCont{ \TR{adds} \NT{MethDecl} }
        \concrCont{ \TR{removes} \NT{MethSig} }
        \concrCont{ \TR{modifies} \NT{MethDecl} }
        \medskip
        \\
        
        \NT{DeltaParams}
        \concrDefn{ \TRS{(}\NT{DeltaParam} \MANYG{\TRS{,} \NT{DeltaParams}} \TRS{)}}

        \\
        \NT{DeltaParam}
        \concrDefn{ \NT{Identifier} \MANY{\NT{HasCondition}}}
        \concrCont{ \NT{Type}\ \NT{Identifier} }
        \medskip
        
        \\
        \NT{ModuleAccess}
        \concrDefn{ \TR{uses} \NT{TypeId} \TRS{;}}
        \medskip
        
        \\
        \NT{HasCondition}
        \concrDefn{ \TR{hasField} \NT{FieldDecl} }
        \concrCont{ \TR{hasMethod} \NT{MethSig} }
        \concrCont{ \TR{hasInterface} \NT{TypeId} }
        
    \end{tabular}
    \caption{ABS Grammar: Delta Modules.}
    \label{fig:delta-grammar}
\end{figure}

Figure~\ref{fig:delta-grammar} specifies the ABS syntax related to delta
modeling. The \NT{DeltaDecl} clause specifies the syntax of delta modules,
consisting of an unique identifier, a module access directive, a list of
parameters and a sequence of module modifiers. The \emph{module
access} directive gives the delta access to the namespace of a particular
module. In other words it specifies the ABS module to which the modifications
specified by the delta apply by default. A delta can still apply changes to
several modules by fully qualifying the \NT{TypeName} of module modifiers.

The \NT{ModuleModifier} clause describes the syntax of modifications at the
level of modules. Such a modification can, for example, add a class or interface
declaration, modify an existing class or interface, remove a class or interface,
and also add functions, data types and type synonyms. Class modifications
include the ability to change the interface of a class by adding or removing
items from the class's list of implemented interfaces.
The \NT{InterfaceModifiers} clause describes how to modify existing interface
declarations, either by adding new or removing existing method signatures.

The \NT{Modifier} clause specifies the modifications that can occur within a
class or interface body. These include adding and removing fields and methods,
and modifying methods, which amounts to replacing a method implementation with a
new one, while enabling the original method to be called using the \TR{original}
keyword.
The aim of \absinline{original} is to enable the method being replaced to be
called from the delta module that replaces it. This is implemented by renaming
the original method, and replacing the call to \absinline{original} with a call
to the renamed method. 

A method can be replaced multiple times by applying a succession of deltas. To
call a specific version of such a method, a \emph{targeted} \absinline{original}
call can be used. The target specifies hereby the delta or the core program.

% This gives the user a tighter control over the program's behaviour. It also
% makes the code less flexible because calling \absinline{original} on a
% particular target delta introduces the assumption that the target delta has been
% already applied. Such a dependency could be invalidated for instance by changes
% to the SPL configuration, which dictates which deltas should be applied when
% certain features are selected.

\section{Object-oriented modifiers}
To modify an object-oriented ABS program, delta modules support adding new
classes and removing existing ones. Existing classes can be also modified by
adding new methods and also removing or modifying existing methods.
Deltas can also add new interface declarations, remove existing interface
declarations, and modify interface declarations by adding or removing
operations. Furthermore, deltas can change the interface of a class by adding or
removing interfaces from the class's list of implemented interfaces. Lastly,
delta modules can introduce new fields to classes and remove existing fields.

\subsection{Interfaces}
Deltas can introduce new interface declarations and remove or modify existing
interface declarations. The syntax is illustrated by the following examples.
\begin{abscode}
delta D1;
adds interface MyModule.I { Unit foo(); }

delta D2; 
uses MyModule;
removes interface I;

delta D3;
uses MyModule;
modifies interface I { removes Unit foo(); adds Unit bar(); }
\end{abscode}


\subsection{Classes}
Deltas can introduce new classes and remove existing classes. The syntax is
illustrated by the following examples.
\begin{abscode}
delta D1;
adds class MyModule.DataBase(Map<Filename,File> db) implements DB {...}

delta D2; 
uses MyModule;
removes class Node();
\end{abscode}

The first delta \absinline{D1} above declares a new class \absinline{DataBase} inside the
module \absinline{MyModule}. Delta \absinline{D2} removes the class \absinline{Node} from the same
module.
Specifying to which module such code modifications apply can be done in two
ways. First, as exemplified by delta \absinline{D1}, the class name can be qualified with a module name.
An alternative way is to include a \absinline{uses <Module Name>} clause at the
beginning of the delta module, which \emph{opens} a module so that names don't
need to be qualified. When a delta specifies modifications to a single module,
this method is more concise. When a delta specifies modifications across
multiple modules, it is more convenient to qualify each class modifier with a module
name. Using both methods together is also possible, in which case unqualified
class names will refer to classes defined inside the \emph{used} module.

Deltas can also \emph{modify} existing classes by adding new methods and
removing or modifying methods; by adding or removing fields; and by manipulating
the list of interfaces that the class implements. These operations are illustrated in the
following sections.

\subsection{Methods}
\label{sec:class methods}
Methods can be added, removed or modified from within deltas.
The following example shows a delta module designed to modify the behaviour of
the class \absinline{Greeter} by modifying its \absinline{sayHello} method.
The class is assumed to have been declared in the core program inside the
\absinline{Hello} module.

\begin{abscode}
delta Nl;
uses Hello;
modifies class Greeter {
    modifies String sayHello() {
        return "Hallo wereld";
    }
}
\end{abscode}
%
The above \absinline{Nl} delta module applies its changes to the core ABS module
\absinline{Hello}, as specified by the \absinline{uses} clause. It provides a new
implementation for the method \absinline{sayHello()} in class
\absinline{Greeter} by declaring a so-called method \emph{modifier}. The method modifier
is introduced by the \absinline{modifies} keyword and followed by the method
signature and a block of code providing the method's new implementation.
 
Adding entirely new methods is also supported using the \absinline{adds} keyword
followed by the method signature and its implementation.
Similarly, it is possible to remove methods from classes using
\absinline{removes} followed by the method signature, as shown in the following.
\begin{abscode}
delta D;
modifies class M.Foo {
    adds Int bar() { return 17; }
    removes Unit moo();
}
\end{abscode}

% \subsubsection{Delta parameters}
% \label{sec:delta parameters}
% Deltas take an optional list of parameters. These are used to pass on
% configuration information defined in the product selection (cf.
% Section~\ref{sec:product selection}) down to the implementation level. Product
% declarations assign a boolean value to each feature (true if selected, false
% otherwise), and a value to each feature attribute (of features that are
% selected). Delta parameters must be immutable objects, such as booleans,
% integers, or strings.
% 
% In the example below, any occurrence of the integer variable \absinline{times}
% inside the delta module is replaced with the concrete value of the feature
% attribute \absinline{Repeat.times} (from the ``Hello World'' example in
% Section~\ref{sec:feature model}) upon delta application. The connection between
% feature attributes and delta parameters is defined in the SPL configuration, 
% which is described in Section~\ref{ch:spl configuration}.
% \begin{abscode}
% delta Rpt (Int times);
% modifies class Hello.Greeter {
%     modifies String sayHello() {
%         String result = "";
%         Int i = 0;
%         while (i < times) {
%             String orig = original();
%             result = result + " " + orig;
%             i = i + 1;
%         }
%         return result;
%     }
% }
% \end{abscode}
% For example, when building the product ``\absinline{P2(Dutch, Repeat\{times=3\})}'',
% the \absinline{times} variable inside the \absinline{while} loops' predicate
% is replaced with the integer constant ``\absinline{3}''.
% 
% The boolean values of features can be accessed in similar fashion, as shown in
% the example below. Here, a delta adds three fields to class \absinline{C}, which
% are assigned boolean constants that reflect whether the features to which they
% are connected are selected or not. In this way, one can easily write code that
% reflects on the feature configuration.
% \begin{abscode}
% delta D (Bool a, Bool b, Bool c);
% modifies class C {
%     adds Bool featureA = a;
%     adds Bool featureB = b;
%     adds Bool featureC = c;
% }
% \end{abscode}

\subsubsection{Calling \absinline{original}}
%A notable artifact in the above example is the statement \absinline{String orig = original()}. 
Calling \absinline{original} from within a method modifier body makes it
possible to access the method's previous behaviour, that is, the behaviour
implemented in the previously applied delta or in the core. This is similar to
calling \absinline{super} to access the superclass behaviour of a method in a language
with class inheritance such as Java. An \absinline{original} call has to supply a list
of arguments that conforms with the original method's list of formal parameters.

\subsubsection{Targeted \absinline{original} calls}
Original calls can be targeted towards a given delta by prefixing the call with
the name of the delta, or towards the core ABS code by using the keyword
\absinline{core}:
\begin{abscode}
core.original(params);
Delta.original(params);
\end{abscode}

Regular (untargeted) original calls invoke the method behaviour defined by the previously applied delta.
% current at the time before application of the delta where the call is issued. 
For example, if a method \absinline{m} is defined in the core, and then a set of deltas
\absinline{D1..D3}, which each modify \absinline{m}, are applied in sequence,
then calling \absinline{original} from within \absinline{m}'s modifier in
\absinline{D3} will run the version of \absinline{m} defined in \absinline{D2}.
With a targeted call, one can access any version of \absinline{m}, that is, the
versions defined in \absinline{D2}, \absinline{D1} and in the core.

This allows a tighter control of which code is actually executed
when calling \absinline{original}. As the order of delta application is often
not uniquely defined, it is not always determinable which behaviour will be invoked upon
calling \absinline{original}. With a targeted original call, the user can
specify exactly which code to execute and even invoke multiple versions of a
method. This, of course, implies that the target delta has been applied already;
otherwise the compiler will indicate an error.

\begin{abscode}
module M;
class C {
    String m(String s) { return(s) };
}
delta D1;
modifies M.C {
    modifies String m(String s) { return prefix + original(s); }
}
delta D2;
modifies M.C {
    modifies String m(String s) { return original(s) + suffix; }
}
delta Resolve;
modifies M.C {
    modifies String m(String s) { return prefix + core.original(s) + suffix; }
}
\end{abscode}

Consider the above example. \absinline{D1} and \absinline{D2} both modify method \absinline{m} in
different, non-compatible ways. We say that these two deltas are in conflict.
Assume that \absinline{D1} and \absinline{D2} can be applied in any order, and that delta
\absinline{Resolve} has to be applied after \absinline{D1} and \absinline{D2}. By calling
\absinline{original} from within \absinline{Resolve}, we cannot be sure which version of
\absinline{m} will actually be invoked: this depends on whether \absinline{D1} or \absinline{D2}
has been applied last. By targeting the original call towards a specific delta,
we can control the behaviour precisely, and resolve the conflict in a meaningful
way.

Targeted original calls were required for the implementation of the delta
modelling workflow (DMW)~\cite{Helvensteijn12,HelvensteijnMW12}, which is
described in more detail in Deliverable 5.3~\cite{d5.3}. The DMW describes a
process of applying delta modelling to obtain a model of a software product line
that is globally unambiguous and complete. A focus of DMW is the systematic
reconciliation of conflicting feature functionality.


\subsection{Class interfaces}
A delta module can change the list of interfaces that a class implements.
Adding or dropping interfaces from that list is achieved using the familiar
\absinline{removes} and \absinline{adds} keywords.

The following example shows a core ABS program defining a \absinline{Logger}
class that implements the \absinline{Output} interface. It further declares a
delta module that modifies the \absinline{Logger} class such that it
implements a different interface. This new  \absinline{IO} interface is 
introduced in the same delta.
%
\begin{abscode}
module M;
interface Input { String read(); }
interface Output { Unit write(String s); }
class Logger implements Output {
    Unit write(String s) {...}
}

delta IO;
adds interface IO extends Input, Output {}
modifies class Logger adds IO removes Output {
    adds String read() {...}
}
\end{abscode}

\subsection{Fields}
In addition to modifying object behaviour, ABS allows adding or removing fields.
New fields are introduced by the \absinline{adds} keyword followed by the
field's type, name, and an optional value assignment. Similarly, fields can be
removed using the \absinline{removes} keyword. The following example demonstrates this.
\begin{abscode}
delta D;
modifies class M.Foo {
    adds List<Item> items;
    adds Int itemsCount = items.size();
    removes String name;
}
\end{abscode}



\section{Functional modifiers}
Functional program elements can also be modified from within deltas.
ABS supports the addition of functions, data types and type synonyms,
and the modification of data types and type synonyms.  Qualifying
functional elements with the module name is currently unsupported,
therefore when adding functional elements, a \absinline{uses} clause has
to be specified.

\subsection{Functions}
Example of adding a function.

\begin{abscode}
delta MyDelta;
uses MyModule;
adds def Int min(Int a, Int b) = case a < b { True => a; False => b; };
\end{abscode}

\subsection{Data types}
\label{sec:data types}
Example of adding a data type.
\begin{abscode}
adds data Schedule = Schedule(
    String schedname,
    List<Item> items,
    Int sched,
    Deadline dline) | NoSchedule;
\end{abscode}

\noindent
Example of modifying a data type.
\begin{abscode}
modifies data Schedule = Schedule(
    String schedname,
    List<Item> items,
    Int sched,
    Deadline dline) | NoSchedule(String reason);
\end{abscode}
When modifying a datatype, the given constructors supersede the previous
list of constructors.

\subsection{Type synonyms}
Example of adding a type synonym.
\begin{abscode}
adds type ClientId = Int;
\end{abscode}

\noindent
Example of modifying a type synonym.
\begin{abscode}
modifies type ClientId = String;
\end{abscode}


\section{Module modifiers}
Deltas support, to some extent, modifications to ABS's module system.

\subsection{Imports and Exports}
The addition of (qualified and unqualified) \absinline{import}s and \absinline{export}s to modules
(cf. Chapter~\ref{sec:modules}) is supported, as shown in the following examples.

\begin{abscode}
delta D1;
uses Drinks;
adds export Drink, Milk;

delta D2;
uses Bar;
adds export *;
adds import Drinks.Milk;

delta D3;
uses MyModule;
adds import * from Bar;
\end{abscode}
%
The \absinline{adds import} and \absinline{adds export} directives apply to the
module defined by the \absinline{uses} statement.


\section{Unsupported modifications}

While delta modelling supports a broad range of ways to modify an ABS model, not
all ABS program entities are modifiable. These unsupported modifications are
listed here for completeness. While these modifications could be easily
specified and implemented, we opted not to overload the language with features
that have not been regarded as necessary in practice.
 
\begin{description}
    \item[Class parameters and init block] Deltas currently do not support the 
    modification of class parameter lists or class init blocks.

  \item[Functional program elements] Deltas currently only support
    adding functions, and adding and modifying data types and type
    synonyms. Removal is not supported.

    \item[Modules] Deltas currently do not support adding new modules or removing modules.
    
    \item[Imports and Exports]While deltas do support the addition of \absinline{import} and
    \absinline{export} statements to modules, they do not support their modification or removal.
    
    \item[Main block] Deltas currently do not support the modification of the program's main block.

\end{description}

% Local Variables:
% TeX-master: "absrefmanual"
% End:
