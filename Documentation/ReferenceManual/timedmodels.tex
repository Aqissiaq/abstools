\chapter{Timed Models}
\label{ch:timedmodels}

ABS models can be augmented with timing information and their timed execution simulated on the Maude backend.  The timed Maude backend adds a model of a clock counting up from 0.  This section describes language features related to this clock.  Most of these features are implemented on the Maude backend and ignored on the Java backend.

\section{Language Elements}

\subsection{Time}

The current value of the clock can be accessed with the expression \absinline{now()}, which returns a value of type \absinline{Time}.  

\begin{absexample} 
data Time = Time(Rat timeValue);  // this is part of the ABS standard library
Time t = now();
\end{absexample}

\subsection{Advancing time in the COG}

The \absinline{duration(min, max)} statement causes execution on the
current COG to be blocked for at least \absinline{min} and at most
\absinline{max} time units.  This is used to simulate methods taking some
amount of time for execution.  (In the Java backend, time passes on
its own, so the \absinline{duration} statement does nothing there.)

\begin{absexample}
// Time will advance between 3 and 5 units during execution of m
Unit m() {
  duration(3, 5);
}
\end{absexample}

\subsection{Advancing time in a process}

The statement \absinline{await duration(min, max)} causes the running
process to be suspended for at least \absinline{min} time units.  The
difference to the \absinline{duration} statement is that other processes in
the same COG are allowed to run while the process is suspended.
Similar to the \absinline{duration} statement, this is for simulation
purposes, so \absinline{await duration} statements do not suspend the
process in the Java backend.


\begin{absexample}
// Method m will return after at least 3 time units
Unit m() {
  await duration(3, 5);
}
\end{absexample}

\subsection{Expressing deadlines}

The deadline (remaining time to execute) for a process can be
accessed via the expression \absinline{deadline()}, which returns a value of
type \absinline{Duration}.  Deadlines are relevant for timed Maude
simulations; a process can detect whether it missed its deadline by
inspecting the return value of \absinline{deadline()}.

\begin{absexample}
// This datatype is part of the ABS standard library
data Duration = Duration(Rat durationValue) | InfDuration;

Duration time_to_complete = deadline(); // Can be infinite
if (deadline() == Duration(0)) {    // We missed the deadline ...
}
\end{absexample}

Deadlines are set at process creation time, i.e. at the calling site.
Synchronous and asynchronous method calls can be decorated with
\absinline{[Deadline: e]} annotations; the default deadline is
\absinline{InfDuration}.

\begin{absexample}
// Give m 17 time units to execute
[Deadline: Duration(17)] o!m();

// Do not give a deadline; this is the default behavior
[Deadline: InfDeadline] o.n();
\end{absexample}

On the Java backend, at the moment deadline annotations are ignored
and \absinline{deadline()} always returns \absinline{InfDuration}.
