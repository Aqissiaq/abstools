\chapter{Algebraic Data Types}
\label{sec:datatypes}
\emph{Algebraic Data Types} make it possible to describe data in an immutable way.
In contrast to objects, values of data types do not have an identify and cannot be mutated.
This makes reasoning about data types much simpler than about objects.
Data types are built by using \emph{Data Type Constructors} (or \emph{constructors} for short), which describe the possible values of a data type.

\begin{abssyntax}
\NT{DataTypeDecl}   \defn \TR{data}\ \NT{TypeId}\ \OPT{\NT{TypeParams}}\ \OPTG{\TR{=}\ \NT{DataConstrList}}\ \TRS{;} \\
\NT{TypeParams}     \defn \TRS{<}\ \NT{TypeId}\ \MANYG{\TRS{,}\ \NT{TypeId}}\ \TRS{>}\\
\NT{DataConstrList} \defn \NT{DataConstr}\ \manyg{\TRS{|}\ \NT{DataConstr}}\\
\NT{DataConstr}     \defn \NT{TypeId}\ \OPTG{\TRS{(} \OPT{\NT{TypeList}} \TRS{)}}
\end{abssyntax}

\begin{absexample}
data IntList = NoInt | Cons(Int, IntList);
data Bool = True | False;
\end{absexample}

\section{Parametric Data Types}
\label{sec:parametric-datatypes}

\emph{Parametric Data Types} are useful to define general-purpose
data types, such as lists, sets or maps.  Parametric data types are
declared like normal data types but have an additional \emph{type
  parameter} section inside broken brackets (\texttt{< >}) after the
data type name.

\begin{absexample}
data List<A> = Nil | Cons(A, List<A>);
\end{absexample}

\section{Predefined Data Types}
\label{sec:predefined-datatypes}

The following data types are predefined:
\begin{itemize}
\item \absinline{data Bool = True | False;}. The boolean type with constructors \texttt{True} and \texttt{False} and the usual Boolean infix and prefix operators.
\item \absinline{data Unit = Unit;}. The unit type with only one constructor \texttt{Unit} (for methods without return values).
\item \absinline{data Int;}. An arbitrary integer ($\mathbb{Z}$) for which values are constructed by using integer literals and arithmetic expressions.
\item \absinline{data Rat;}. A rational number ($\mathbb{Q}$).  Rational values are obtained via the division (\absinline{/}) operator and have arbitrary precision.  Assigning rational values to variables of type \absinline{Int}, either explicitly or implicitly by passing them to a function or method expecting an integer, rounds towards zero.
\item \absinline{data String;}. A string for which values are constructed by using string literals and operators.
\item \absinline{data Fut<T>;}. Representing a future. A future cannot be explicitly constructed, but it is the result of an asynchronous method call. The value of a future can only be obtained by using the \texttt{get} expression (Sec.~\ref{sec:getexpr}).
\item \absinline{data List<A> = Nil | Cons(A, List<A>)}, with constructors \absinline{Nil} and
  \absinline{Cons(A, List<A>)}.  This predefined data type is used
  for implementing arbitrary n-ary constructors (see below).
\end{itemize}

A complete list of predefined data types is contained in
Appendix~\ref{ch:absstdlib} which lists the ABS Standard Library.

\section{N-ary Constructors}
\label{sec:n-ary-constructors}

For data types of arbitrary size, like lists, maps and sets, it is
undesirable having to write them down in the form of nested constructor
expressions.  For this purpose, ABS provides a special syntax for
\emph{n-ary constructors}, which are transformed into constructor
expressions via a user-supplied function.

\begin{absexample}
data Set<A> = EmptySet | Insert(A, Set<A>);
def Set<A> set<A>(List<A> l) = 
   case l {
      Nil => EmptySet;
      Cons(hd, tl) => Insert(hd, set(tl));
   } ;

{
  Set<Int> s = set[1, 2, 3];
}
\end{absexample}
An expression \grsh{type[parameters*]} is transformed into a literal by
handing it to a function named \grsh{type} which takes one parameter of
type \grsh{List} and returns an expression of type \grsh{Type}.  (It is
desirable, although not currently enforced, that \grsh{type} and
\grsh{Type} are the same word, just with different capitalization.)

\section{Abstract Data Types}
\label{sec:abstract data types}
Using the module system (cf.~Sec.~\ref{sec:modules}) it is possible to define \emph{abstract data types}. For an abstract data type, only the functions that operate on them are known to the client, but not its constructors. This can be easily realized in ABS by putting such a data type in its own module and by only exporting the data type and its functions, without exporting the constructors.

\section{Exceptions}
\label{sec:exceptions}

Exceptions (see also Section~\ref{sec:error-handling}) are modeled as abstract
datatypes.  All exceptions are of type \grsh{ABS.StdLib.Exception} and are
declared with the keyword \grsh{exception}.  Exceptions have a unique name and
an optional list of parameters.

\begin{abssyntax}
\NT{ExceptionDecl}   \defn \TR{exception}\ \NT{TypeId}\ \OPTG{\TRS{(} \OPT{\NT{TypeList}} \TRS{)}}
\end{abssyntax}

% Local Variables:
% TeX-master: "absrefmanual.tex"
% End:
