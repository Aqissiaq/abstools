\chapter{Classes and Interfaces}
\label{sec:classandint}
Objects in ABS are built from \emph{classes}, which implement
\emph{interfaces}. Only interfaces can be used as types for objects in
ABS.

\section{Interfaces}
\label{sec:interfacedecl}
Interfaces in ABS are similar to interfaces in Java.
They have a name, which defines a nominal type, and 
they can \emph{extend} arbitrary many other interfaces.
The interface body consists of a list of method signature declarations. 
Method names start with a lowercase letter.

\begin{abssyntax}
\NT{InterfaceDecl} \defn \TR{interface}\ \NT{TypeId}\ \OPTG{\TR{extends}\
  \NT{TypeName}\ \MANYG{\TRS{,}\ \NT{TypeName}}}\
  \TRS{\{}\ \MANY{\NT{MethSig}}\ \TRS{\}}\\
%\NT{MethSigList}   \defn \MANY{\NT{MethSig}}\\
\NT{MethSig}       \defn \NT{Type}\ \NT{Identifier}\ \TRS{(}\ \OPT{\NT{ParamList}}\ \TRS{)}\ \TRS{;}\\
\NT{ParamList}     \defn \NT{Param}\ \MANYG{\TRS{,}\ \NT{Param}}\\
\NT{Param}         \defn \NT{Type}\ \NT{Identifier}
\end{abssyntax}

%\begin{absexample}
%interface Foo {
%   Unit m(Bool b, Int i);
%}
%
%interface Bar extends Foo {
%   Bool n();    
%}
%\end{absexample}

The interfaces in the example below represent a database system, providing
functionality to store and retrieve files, and a node of a peer-to-peer file
sharing system. Each node of a peer-to-peer system plays both the role of a
server and a client. The data types are defined in the ABS standard library,
included in Appendix~\ref{ch:absstdlib}, and the remainder types are type synonyms
included in Section~\ref{sec:typesynonyms}.

\begin{absexample}
interface DB {
  File getFile(Filename fId);
  Int getLength(Filename fId);
  Unit storeFile(Filename fId, File file);
  Filenames listFiles();
}

interface Client {
  List<Pair<Server,Filenames>> availFiles(List<Server> sList);
  Unit reqFile(Server sId, Filename fId);
}

interface Server {
  Filenames inquire();
  Int getLength(Filename fId);
  Packet getPack(Filename fId, Int pNbr);
}

interface Peer extends Client, Server {
  List<Server> getNeighbors();
}
\end{absexample}

\section{Classes}
Like in typical class-based languages, classes in ABS 
are used to create objects. 
Classes can implement an arbitrary number of interfaces.
ABS does not support class inheritance, as code reuse in ABS is realized by delta modules (see Chapter~\ref{ch:deltas}).
Classes do not have constructors in ABS but instead have
\emph{class parameters} and an optional \emph{init block}.
Class parameters actually define additional fields of the 
class that can be used like any other declared field.
\newpage

\begin{abssyntax}
\NT{ClassDecl}     \defn \TR{\NT{class}}\ \NT{TypeId}\ \OPTG{\TRS{(} \NT{ParamList} \TRS{)}}\ \OPTG{\TR{implements}\ \NT{TypeName}\ \MANYG{\TRS{,}\ \NT{TypeName}}} 
\\&&
                  \TRS{\{}\ \OPT{\NT{FieldDeclList}}\ \OPT{\NT{Block}}\ \OPT{\NT{MethDeclList}}\ \TRS{\}}\\  
\NT{FieldDeclList} \defn \NT{FieldDecl}\ \MANYG{\TRS{,}\ \NT{FieldDecl}}\\
\NT{FieldDecl}     \defn \NT{TypeId}\ \NT{Identifier}\ \OPTG{\TRS{=}\ \NT{PureExp}}\ \TRS{;}\\
\NT{MethDeclList}  \defn \NT{MethDecl}\ \MANYG{\TRS{,}\ \NT{MethDecl}}\\
\NT{MethDecl}      \defn \NT{Type}\ \NT{Identifier}\ \TRS{(} \NT{ParamList} \TRS{)}\ \NT{Block}
\end{abssyntax}

%\begin{absexample}
%class Foo(Bool b, Int i) implements Bar, Baz {
%   Int j = 5;
%   Bar b;
%
%   {   
%     j = i;
%   }  
%
%   Bool m() { 
%     return True;
%   }
%}
%\end{absexample}

We continue with the peer-to-peer example with an implementation of
the \absinline{DB} interface, and the signature of a node that
implements the \absinline{Peer} interface.
%The remainder types, 
%\absinline{Filename}, \absinline{Filenames}, \absinline{File}, \absinline{Packet}, and \absinline{Catalog}, are defined as type synonyms, that we will introduce 
%
%... sec:typesynonyms

\begin{absexample}
class DataBase(Map<Filename,File> db) implements DB {
  File getFile(Filename fId) {
      return lookup(db, fId);
  }
  
  Int getLength(Filename fId){
      return length(lookup(db, fId));
  }
  
  Unit storeFile(Filename fId, File file) {
      db = insert(Pair(fId,file), db);
  }
  
  Filenames listFiles() {
      return keys(db);
  }
}

class Node(DB db, Peer admin, Filename file) implements Peer {
  Catalog catalog;
  List<Server> myNeighbors;

  // implementation...
}
\end{absexample}

\subsection{Active Classes}
\label{sec:active-classes}

A class can be \emph{active} or \emph{passive}.  Active classes start an
activity on their own upon creation. Passive classes only react to
incoming method calls. A class is active if and only if it has a \emph{run method}:
\begin{absexample}
Unit run() {
  // active behavior ...
}
\end{absexample}
The run method is activated after object initialization.  
