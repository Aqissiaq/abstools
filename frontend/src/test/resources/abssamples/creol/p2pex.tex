\documentclass[handout]
{beamer}
\usepackage[english]{babel}
\usepackage[utf8]{inputenc}
\usepackage{helvet}
\usepackage{amssymb}
\usepackage{stmaryrd}
\usepackage[T1]{fontenc}
\usepackage{creol}

%%% from creol
\usepackage{textcomp}
\usepackage{pgf,pgfpages}
%\pgfpagesuselayout{4 on 1}[a4paper,landscape,border shrink=5mm] \nofiles
%\usetheme{Malmoe}
\useoutertheme{infolines}

\newcommand{\creolX}{\mbox{Creol}}
%\newcommand{\creol}{\mbox{\creolX} }

% Metacommands
\newcommand{\TYPE}[1]{\textsf{\small #1}}
\newcommand{\PLCOMM}[1]{\textbf{#1}}
\newcommand{\GUARD}[1]{\mathit{#1}}

%REWRITE-PIL:
\newcommand{\rew}{\mbox{$\;\longrightarrow\;$}}
\newcommand{\Lab}{\textsl{Lcnt}}
\newcommand{\Next}{\textsl{next}}
\newcommand{\Tokens}{\textsl{Ocnt}}
\newcommand{\NEWID}{(C;n)}
\newcommand{\Cl}[2]{ \langle #1 \!:\! \id{Class}\: |\: #2 \rangle}
\newcommand{\Ob}[2]{ \langle #1 \: |\: #2 \rangle}
\newcommand{\Qu}[2]{ \langle #1 \!:\! \id{Qu}\: |\: #2 \rangle}
\newcommand{\sig}{\id{Sig}}
\newcommand{\coint}{\id{Co}}
\newcommand{\key}[1]{\textbf{#1}}

\newcommand{\wait}{\GUARD{wait}}
% \renewcommand{\*}[1]{{\textsc{#1}}}
\renewcommand{\*}[1]{\textsc{\MakeUppercase{#1}}}
\newcommand{\findAttr}{\textsl{getAttr}}
\newcommand{\foundAttr}{\textsl{gotAttr}}
%\newcommand{\merge}{\mbox{$|\!|\!|$} }

\newcommand{\true}{\textsf{\small true}}
\newcommand{\false}{\textsf{\small false}}

\newcommand{\guardsX}{\TYPE{Guard}}
\newcommand{\guards}{{\guardsX} }
\newcommand{\boolexprX}{\TYPE{BoolExpr}}
\newcommand{\boolexpr}{{\boolexprX} }
\newcommand{\labelsX}{\TYPE{Label}}
\newcommand{\labels}{{\labelsX} }
\newcommand{\mtdcallX}{\TYPE{MtdCall}}
\newcommand{\mtdcall}{{\mtdcallX} }
\newcommand{\comX}{\TYPE{Com}}
\newcommand{\com}{{\comX} }
\newcommand{\comlistX}{\TYPE{ComList}}
\newcommand{\comlist}{{\comlistX} }
\newcommand{\exprlistX}{\TYPE{ExprList}}
\newcommand{\exprlist}{{\exprlistX} }
\newcommand{\mtdX}{\TYPE{Mtd}}
\newcommand{\mtd}{{\mtdX} }
\newcommand{\objectexprX}{\TYPE{ObjExpr}}
\newcommand{\objectexpr}{{\objectexprX} }
\newcommand{\dataX}{\TYPE{Data}}
\newcommand{\data}{{\dataX} }

% Variables
\newcommand{\varX}{\TYPE{Var}}
\newcommand{\var}{{\varX} }
\newcommand{\varlistX}{\TYPE{VarList}}
\newcommand{\varlist}{{\varlistX} }

\newcommand{\id}[1]{\textsl{#1}}
\newcommand{\TO}{~\textbf{to}~}
\newcommand{\innrykk}{\mbox{\hspace{3mm}}}

% The Programming Language
\newcommand{\SKIP}{\PLCOMM{skip}}
\newcommand{\RELEASE}{\PLCOMM{release}}
\newcommand{\NEW}{\PLCOMM{new}}
\newcommand{\GET}{\PLCOMM{get}}
\newcommand{\CALL}{\PLCOMM{call}}
\newcommand{\REPLY}{\PLCOMM{reply}}
\newcommand{\IF}{\PLCOMM{if}}
\newcommand{\FI}{\PLCOMM{fi}}
\newcommand{\DO}{\PLCOMM{do}}
\newcommand{\OD}{\PLCOMM{od}}
\newcommand{\OP}{\PLCOMM{op}}
\newcommand{\IN}{\PLCOMM{in}}
\newcommand{\OUT}{\PLCOMM{out}}
\newcommand{\THEN}{\PLCOMM{then}}
\newcommand{\ELSE}{\PLCOMM{else}}
\newcommand{\WHILE}{\PLCOMM{while}}
\newcommand{\AWAIT}{\PLCOMM{await}}
\newcommand{\BEGIN}{\PLCOMM{begin}}
\newcommand{\END}{\PLCOMM{end}}
\newcommand{\VAR}{\PLCOMM{var}}
\newcommand{\WITH}{\PLCOMM{with}}
\newcommand{\ANY}{\textsl{Any}}
\newcommand{\others}{\PLCOMM{others}}
\newcommand{\assign}{assign} 
\newcommand{\getNames}[1]{#1}
% \newcommand{\setsize}[1]{{\small #1}}
\newcommand{\CLASS}{\PLCOMM{class}}
\newcommand{\CLASSBEGIN}{\PLCOMM{begin}}
\newcommand{\CLASSEND}{\PLCOMM{end}}
\newcommand{\INHERITS}{\PLCOMM{inherits}}
\newcommand{\INTERFACE}{\PLCOMM{interface}}
\newcommand{\IMPLEMENTS}{\PLCOMM{implements}}
\newcommand{\gr}{\id{gr}}


\newcommand{\Empty}{\varepsilon}
\newcommand{\Constructor}{;}
\def\d{~\#~}


\usetheme{Boadilla}%{Malmoe}%
\mode<presentation>
%\mode<handout>
\definecolor{code}{rgb}{0,0.7,0}

\newcommand{\send}[3]{#1\!\uparrow\! #2\!\!:\!\!#3}%{\mbox{$#1\!\uparrow\! #3\!:\! #2$}}%{\mbox{$#1\!:\! #3\!\uparrow\! #2$}}%{\mbox{$#1\!:\! #3\!\pil\! #2$}}
\newcommand{\rec}[3]{#1\!\downarrow\! #2\!\!:\!\!#3}%{\mbox{$#1\!:\! #3\!\downarrow\! #2$}}%
%{\mbox{$#1\!:\! #3\!\downarrow\! #2$}}%{\mbox{$#1\!:\! #3\!\leftarrow\! #2$}}
\newcommand{\sendrec}[3]{{\send{#1}{#2}{#3}}, \rec{#1}{#2}{#3}}
\newcommand{\SENDREC}[3]{\mbox{$#1\!\updownarrow\!#2:#3$}}%
%{\mbox{$#1\!:\! #3\!:\! #2$}}%{\mbox{$#1\!:\!#3\!\updownarrow\! #2$}}
\newcommand{\mboxSend}[3]{\mbox{$\send{#1}{#2}{#3}$}}% for use in alltt
\newcommand{\mboxRec}[3]{\mbox{$\rec{#1}{#2}{#3}$}}% for use in alltt
\newcommand{\Merk}[1]{\blue{\bf #1}}
\newcommand{\Emph}[1]{\red{\emph{#1}}}
\newcommand{\ignore}[1]{}

\begin{document}

\lstset{language=Creol,columns=flexible}


% \begin{frame}
% \frametitle{Example: A Peer To Peer Network}
% \begin{columns}[t]
% \begin{column}{65mm}
% \quad\\[3mm]
% \begin{itemize}
% \item Consider a P2P file sharing network
% \item Nodes in the network exchange files
% \item File transfer by many packages
% \item Each node may be involved in several uploads/downloads
%   simultaneously
% \item Nodes may appear, disappear, and reappear
% \uncover<2|handout:1>{\item Interfaces describe the different aspects of
%   the system:\\ \red{\bf DB}, \red{\bf Client}, \red{\bf Server}}
% \end{itemize}
% \end{column}
% \begin{column}{61mm}
% \quad\\[-7mm]
% \only<1|handout:0>{\includegraphics[width=60mm]{Figures/p2p_1}}%
% \only<2|handout:1>{\includegraphics[width=60mm]{Figures/p2p_2}}%
% \end{column}
% \end{columns}
% \end{frame}

\begin{frame}
P2P Example: 

Interfaces: DB, Server, Client
 
Classes: DataBase, Node

Note:
could also add interfaces P2P inheriting both Server and Client
and being implemented by Node.
\end{frame}

\begin{frame}
\frametitle{P2P Example: The Data Base}
The database stores file locally for a \red{\bf Server} node\\ 
and provides a \red{\bf DB} interface
\begin{itemize}
\item getFile: retrieve file from database
\item storeFile: store file in database
\item listFile: list available files in database
\item getLength: get length of given file
\end{itemize}

\bigskip
\pause

\begin{small}
\INTERFACE\ \red{\bf DB}\\
\BEGIN\\
\innrykk \WITH\ \red{\bf Server}\\
\innrykk \innrykk \OP\ getFile(\IN\ fId: String; \OUT\ file: List[Package])\\
\innrykk \innrykk \OP\ getLength(\IN\ fId: String; \OUT\ length: Int)\\
\innrykk \innrykk \OP\ storeFile(\IN\ fId: String, file: List[Package])\\
\innrykk \innrykk \OP\ listFiles(\OUT\ fSet: Set[String])\\
\END
\end{small}
\end{frame}


\begin{frame}
\frametitle{P2P Example: The Client}
The \red{\bf Client} interface of the nodes provides services to other \red{\bf Clients}
\begin{itemize}
\item availFiles: which files are available in P2P network?
\item reqFile: request a given file from a given server
\end{itemize}

\bigskip
\pause

\begin{small}
\INTERFACE\ \red{\bf Client}\\
\BEGIN\\
\innrykk \WITH\  \red{\bf Client}\\
\innrykk \innrykk \OP\ availFiles (\IN\ sList: List[Server]; \OUT\ files: List[[Server, Set[String]]])\\
\innrykk \innrykk \OP\ reqFile(\IN\ sId: Server, fId: String)\\
\END
\end{small}
\end{frame}

\begin{frame}
\frametitle{P2P Example: The Server}
A \red{\bf Server} interface of the nodes provides services to other \red{\bf Servers}
\begin{itemize}
\item enquire: list available files at the given server
\item getLength: length of a given file in the server
\item getPack: get a part of a given file from the server
\end{itemize}

\bigskip
\pause

\begin{small}
\INTERFACE\ \red{\bf Server}\\
\BEGIN\\
\innrykk \WITH\ \red{\bf Server}\\
\innrykk \innrykk \OP\ enquire(\OUT\ files: Set[String])\\
\innrykk \innrykk \OP\ getLength(\IN\ fId: String; \OUT\ lth: Int)\\
\innrykk \innrykk \OP\ getPack(\IN\ fId: String, pNbr: Int; \OUT\ pack: Package)\\
\END
\end{small}
\end{frame}


\section{P2P Example (Cont)}
\begin{frame}
\frametitle{P2P Example}
\
We now look at how to model the classes \red{\bf DataBase} and \red{\bf Node}.
\end{frame}

\begin{frame}
\frametitle{The Database}

\begin{small}
\INTERFACE\ DB\\
\BEGIN\ \WITH\ Server\\
\innrykk \innrykk \OP\ getFile(\IN\ fId: String; \OUT\ file: List[Package])\\
\innrykk \innrykk \OP\ getLength(\IN\ fId: String; \OUT\ length: Int)\\
\innrykk \innrykk \OP\ storeFile(\IN\ fId: String, file: List[Package])\\
\innrykk \innrykk \OP\ listFiles(\OUT\ fSet: Set[String])\\
\END
\end{small}

\bigskip
\pause

\blue{\bf One possible implementation of the DB interface:}

\medskip
\pause

\begin{small}
\CLASS\ \red{\bf DataBase}(db : Map[String, List[Package]]) \IMPLEMENTS\ DB\\
\BEGIN\ \WITH\ Server\\
\innrykk \innrykk \OP\ \red{\bf getFile}(\IN\ fId: String; \OUT\ file: List[Package])== file := get(db, fId)\\
\innrykk \innrykk \OP\ \red{\bf getLength}(\IN\ fId: String; \OUT\ length: Int)== \\
\innrykk \innrykk \innrykk \innrykk \innrykk \innrykk length := \#(get(db,fId))\\
\innrykk \innrykk \OP\ \red{\bf storeFile}(\IN\ fId: String, file: List[Package])==\\
\innrykk \innrykk \innrykk \innrykk \innrykk \innrykk  db := insert(db, fId, file)\\
\innrykk \innrykk \OP\ \red{\bf listFiles}(\OUT\ fSet: Set[String])== fSet := keys(db)\\
\END
\end{small}

\end{frame}

\begin{frame}
\frametitle{The Nodes}
\begin{small}
\CLASS\ \red{\bf Node}(db: DB, admin:P2P, file:String) \IMPLEMENTS\ Peer\\
\BEGIN\\
\innrykk \VAR\ catalog : List[[Server, Set[String]]];\\
\quad \\
\innrykk \OP\ \red{\bf findServer}(\IN\ fId:String, catalog : List[[Server, Set[String]]]; \\
\innrykk \innrykk \innrykk \innrykk \innrykk \innrykk \innrykk \innrykk \innrykk 
\OUT\ server:Server)==\\
\innrykk \innrykk \innrykk       \IF\ isempty(catalog) \THEN\ server := null \\
\innrykk \innrykk \innrykk \innrykk       \ELSE\ \IF\ (fId \IN\ snd(head(catalog))) \THEN\ server := fst(head(catalog)) \\
\innrykk \innrykk \innrykk \innrykk \innrykk            \ELSE\ findServer(fId, tail(catalog); server) \END\ \END\\
\quad \\
\innrykk \OP\ \red{\bf run} == \\
\innrykk \innrykk     \VAR\ l:Label[List[[Server, Set[String]]]]; \\
\innrykk \innrykk \innrykk     \VAR\ neighbors: List[Server]; \VAR\ server:Server;\\
\innrykk \innrykk \innrykk \innrykk admin.getNeighbors(;neighbors); \\
\innrykk \innrykk \innrykk \innrykk \innrykk     l!this.availFiles(neighbors); \AWAIT\ l?; l?(catalog);\\
\innrykk \innrykk \innrykk \innrykk \innrykk \innrykk     findServer(file,catalog; server); this.reqFile(server,file;)
\end{small}
\end{frame}

\begin{frame}
\blue{\bf Implementing the Server interface}

\bigskip

\begin{small}
\innrykk   \red{\bf \WITH\ Server}\\
\innrykk \innrykk \OP\ \red{\bf enquire}(\OUT\ files: Set[String]) == 
\AWAIT\ db.listFiles(; files)\\
\innrykk \innrykk \OP\ \red{\bf getLength}(\IN\ fId: String; \OUT\ lth: Int) == \AWAIT\  db.getLength(fId; lth)\\
\innrykk \innrykk \OP\ \red{\bf getPack}(\IN\ fId: String, pNbr: Int; \OUT\ pack: Package) ==\\
\innrykk \innrykk \innrykk \VAR\ f: List[Package]; \AWAIT\ db.getFile(fId; f); pack := nth(f, pNbr)\\
\end{small}
\end{frame}

\begin{frame}
\blue{\bf Implementing the Client interface}

\bigskip

\begin{small}
\innrykk  \red{\bf \WITH\ Client}\\
\innrykk \innrykk     \OP\ \red{\bf availFiles} (\IN\ sList: List[Server]; \OUT\ files: List[[Server, Set[String]]]) ==\\
\innrykk \innrykk \innrykk \VAR\ l1: Label[Set[String]]; \VAR\ l2: Label[List[[Server, Set[String]]]] ;\\
\innrykk \innrykk \innrykk \innrykk \VAR\ fList: Set[String];\\
\innrykk \innrykk \innrykk \innrykk \innrykk \IF\ (sList = nil) \THEN\ files := nil\\
\innrykk \innrykk \innrykk \innrykk \innrykk \innrykk \ELSE\ l1!head(sList).enquire(); 
l2!this.availFiles(tail(sList));\\
\innrykk \innrykk \innrykk \innrykk \innrykk \innrykk \innrykk	
\AWAIT\ l1? $\land$ l2?; l1?(fList); l2?(files);\\
\innrykk \innrykk \innrykk \innrykk \innrykk \innrykk \innrykk \innrykk 	
files := files $\vdash$ (head(sList), fList)\ \END\\
\quad \\
\innrykk \innrykk  \OP\ \red{\bf reqFile}(\IN\ sId: Server, fId: String) ==\\
\innrykk \innrykk \innrykk \VAR\ file: List[Package] := nil; \VAR\ pack: Package; \VAR\ lth: Int;\\
\innrykk \innrykk \innrykk \innrykk \AWAIT\ sId.getLength(fId; lth);\\
\innrykk \innrykk \innrykk \innrykk \innrykk \WHILE\ (lth > 0) \DO\ \AWAIT\ sId.getPack(fId, lth; pack);\\
\innrykk \innrykk \innrykk \innrykk \innrykk \innrykk \innrykk \innrykk \innrykk \innrykk \innrykk \innrykk \innrykk \innrykk \innrykk 
file := pack $\dashv$ file; lth := lth - 1\ \END;\\
\innrykk \innrykk \innrykk \innrykk \innrykk \innrykk !db.storeFile(fId, file)\\
\END
\end{small}
\end{frame}


\begin{frame}
\frametitle{Notation of underlying data types and functions}
The predefined \emph{list} type has functions
\begin{description}
\item[{\#}] returning the length of a list
\item[nth]  indexing a list 
%\item[\_[\_]] indexing a list 
\item[{$\dashv$}] appending a list with an element
\item[{$\vdash$}] pre-pending a list with an element.
\item[head] returning the first element of a list. \red{$head(x\vdash l)=x$}.
\item[tail] returning the rest of the list. Thus, \red{$tail(x\vdash l)=l$}.
\end{description}
The predefined type \emph{Map} is a set of associations, here used
for associations
 from strings to list of packages.  Functions:
\begin{description}
\item[get]
is used to
retrieve information from a map for a given key
\item[{insert}] to
insert a new association in a map
\item[{keys}] to get the set of keys
in a map.
\end{description}
The type \blue{$[A,B]$} denotes \emph{pairs} of $A$ and $B$. The
functions \blue{\emph{fst}} and \blue{\emph{snd}} return the first and
second element of a pair, respectively. Thus, \red{$p=(fst(p),snd(p))$}
if $p$ is a pair.
\end{frame}

\end{document}