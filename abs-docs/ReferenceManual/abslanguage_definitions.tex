%%%%%%%%%%%%%%%%%%%%%%%%%%%%%%%%%%%%%%%%%%%%%%%%%%%%%%%%%%%%
% some commonly used commands

\definecolor{darkblue}{rgb}{0,0,0.6}
\definecolor{darkgreen}{rgb}{0,0.6,0}
\definecolor{darkred}{rgb}{0.7,0,0}

\newcommand{\mynote}[1]{
  \ensuremath{\left\langle\right.\!\!\left\langle\right.\!}\bf #1%
  \ensuremath{\!\left.\right\rangle\!\!\left.\right\rangle}}
\newcommand{\DC}[1]{\textcolor{blue}{\mynote{DC: #1}}}
\newcommand{\RM}[1]{\textcolor{darkgreen}{\mynote{RM: #1}}}
\newcommand{\JP}[1]{\textcolor{darkred}{\mynote{JP: #1}}}
\newcommand{\JS}[1]{\textcolor{violet}{\mynote{JS: #1}}}

\newcommand{\muTVL}{\ensuremath{\mu\textrm{TVL}}\xspace}
\newcommand{\fid}{\ensuremath{\mathsf{FID}}\xspace}
\newcommand{\aid}{\ensuremath{\mathsf{AID}}\xspace}
\newcommand{\name}{\ensuremath{\mathsf{Name}}\xspace}

%%%%%%%%%%%%%%%%%%%%%%%%%%%%%%%%%%%%%%%%%%%%%%%%%%%%%%%%%%%%
%% Concrete syntax used in D1.2

\newcommand{\NT}[1]{\textit{#1}}
%\newcommand{\TR}[1]{\texttt{\underline{\raisebox{-2.2pt}{\rule{0pt}{0pt}}#1}}}
\newcommand{\TR}[1]{\ensuremath{\mathtt{#1}}}
\newcommand{\TRS}[1]{\texttt{#1}}
\newcommand{\VAR}[1]{\textsc{#1}}
%\newcommand{\MANY}[1]{\ensuremath{\bnfmany{#1}}}
%\newcommand{\MANYG}[1]{\ensuremath{\bnfmany{(#1)}}}
\newcommand{\MANY}[1]{#1\ensuremath{^*}}
\newcommand{\MANYG}[1]{\ensuremath{(}#1\ensuremath{)^*}}
\newcommand{\PLUS}[1]{\bnfplus{#1}}
\newcommand{\PLUSG}[1]{\bnfplus{(#1)}}
\newcommand{\OPT}[1]{\ensuremath{[}#1\ensuremath{]}}
\newcommand{\OPTG}[1]{\ensuremath{[}#1\ensuremath{]}}
%\raisebox{-2pt}{\rule{0pt}{0pt}

%\newcommand{\defNT}[1]{\NT{#1} &::=&}
\newcommand{\defn}{&::=&}
\newcommand{\defc}{\\&\multicolumn{1}{r}{\ensuremath{|}}&}
\newcommand{\concrDefn}[1]{\defn #1}
\newcommand{\concrCont}[1]{\defc #1}
\newcommand{\concrNewline}[1]{\\&& #1}


\newenvironment{bnfgrammar}
{\begin{tabular}{lrl@{\hspace{3mm}}l}}
{\end{tabular}}

\newenvironment{abssyntax}
{~\\[1ex]\noindent\textbf{Syntax:}\\[1ex]
\begin{bnfgrammar}}
{\end{bnfgrammar}\\[1ex]}

\newenvironment{abstractgrammar}
{\[ \begin{array}[t]{@{}lrl@{\hspace{3mm}}l@{}}}
{\end{array}\]}

%%%%%%%%%%%%%%%%%%%%%%%%%%%%%%%%%%%%%%%%%%%%%%%%%%%%%%%%%%%%
%% Making grammars consistent

\newcommand{\bnfoptional}[1]     {{ [} #1 { ]}\, }
\newcommand{\bnfplus}[1]{#1\ensuremath{^+}}%one or more occurrences, separated by comma
\newcommand{\bnfdef}               {::=}
\newcommand{\bnfbar}               {\mathrel{|}}

\newcommand{\seqof}[1]{\MANY{#1}}
\newcommand{\bnfmany}[1]{\MANY{#1}}
\newcommand{\many}[1]{\MANY{#1}}
\newcommand{\manymath}[1]{\ensuremath{\overline{#1}}}
\newcommand{\kw}[1]{\ensuremath{\mathtt{#1}}}  % same as \TR

\newcommand{\seqofg}[1]{\MANYG{#1}}
\newcommand{\bnfmanyg}[1]{\MANYG{#1}}
\newcommand{\manyg}[1]{\MANYG{#1}}

\newcommand{\grshape}{\slshape}
%\newcommand{\grface}{\texttt}

\newcommand{\gmany}[1]     {\{#1\}}                      %%% 
%\newcommand{\gopt}[1] {\textup{[}{#1}\textup{]}}
\newcommand{\gopt}[1] {[{#1}]}
\newcommand{\gbar}{\ensuremath{|\ }}
\newcommand{\posix}[1]     {\textup{[:\,}{#1}\textup{\,:]}}                      %%% 

%\newcommand{\grface}{\textsl}
%\newcommand{\grface}{\texttt}

\newcommand{\sourcefont}{\ttfamily}
\newcommand{\commentfont}{\slshape\rmfamily\color{black!70}}

%%%%%%%%%%%%%%%%%%%%%%%%%%%%%%%%%%%%%%%%%%%%%%%%%%%%%%%%%%%%

\colorlet{keywordcolor}{blue!50!black}

\lstdefinelanguage{ABS}{
 language=Java,
 deletekeywords={void,long,int,byte,boolean,true,false}, 
 morekeywords={type, data, def, export, import, get, let, local, then, in, await, assert,suspend,module,from,now,duration,deadline,
delta,uses,adds,modifies,removes,original,core,productline,corefeatures,optionalfeatures,after,when,product,hasAttribute,hasMethod},
}

%\def\codesize{\fontsize{9}{10}}

\lstnewenvironment{abs}{%
  \lstset{language=ABS,columns=fullflexible,mathescape=true,%
           keywordstyle=\bf\sffamily,commentstyle=\sl\sffamily,%
           basicstyle=\footnotesize\normalfont\sffamily,inputencoding=latin1, % i would prefer utf8
           extendedchars,xleftmargin=2em,showstringspaces=false}%
  \csname lst@SetFirstLabel\endcsname}
  {\csname lst@SaveFirstLabel\endcsname}


%%%% mTVL
\lstdefinelanguage{MTVL}{keywords=
{root,group,extension,oneof,allof,opt},
  sensitive=true,
  morecomment=[l]{//},
  morestring=[b]"}

%\def\codesize{\fontsize{9}{10}}
\lstnewenvironment{mtvl}{%
 \lstset{language=MTVL,columns=fullflexible,mathescape=true,%
           keywordstyle=\bf\sffamily,commentstyle=\sl\sffamily,%
           basicstyle=\footnotesize\normalfont\sffamily,inputencoding=latin1, % i would prefer utf8
           extendedchars,xleftmargin=2em}%
   \csname lst@SetFirstLabel\endcsname}
  {\csname lst@SaveFirstLabel\endcsname}

%%%% Spec lang

\lstdefinelanguage{SPEC}{keywords={interface,spec,in,out,calls,returns},
  sensitive=true,
  comment=[l]{//},
  morecomment=[s]{/*}{*/},
  morestring=[b]"}
\lstnewenvironment{spec}{%
\lstset{language=SPEC,columns=fullflexible,mathescape=true,%
        keywordstyle=\bf\sffamily,commentstyle=\sl\sffamily,%
        basicstyle=\footnotesize\normalfont\sffamily,inputencoding=latin1, % i would prefer utf8
        extendedchars,xleftmargin=2em}%
  \csname lst@SetFirstLabel\endcsname}
  {\csname lst@SaveFirstLabel\endcsname}

\lstdefinestyle{absgrammar}{
  basicstyle=\ttfamily,
%  moredelim=[is][\bfseries]{'}{'},
%  morecomment=[s][\bfseries]{'}{'},
%  stringstyle=\bfseries,
  morestring=[b]',
  columns = fullflexible,%spaceflexible,%fullflexible,
  columns = fixed,
  fontadjust = true,
  keepspaces = true, % do not summarize spaces
  mathescape={false},
  xleftmargin=1.5em,
 showstringspaces = false,
}

\lstdefinestyle{absnobg} {
 language=ABS,
 basicstyle=\sourcefont\upshape,
 commentstyle=\commentfont,
 keywordstyle=\bfseries\color{keywordcolor}\upshape,
% deletekeywords={static,public,private}
 classoffset=1,
 % standard types:
 morekeywords={Unit,Int, Rat, Bool, List, Set, Pair, Fut, Maybe, String, Triple, Either, Map},
 keywordstyle=\color{violet},
 classoffset=0,
 mathescape={true},
 escapechar={\#},
 columns = fullflexible,%spaceflexible,%fullflexible,
 fontadjust = true,
 keepspaces = true, % do not summarize spaces
 showstringspaces = false,
% inputencoding=utf8,
% numbers=left,
 xleftmargin=1.5em,
framexleftmargin=1em,
framextopmargin=0.5ex,
% breaklines=true,
% lineskip=0pt, 
% abovecaptionskip=-1ex,
% belowcaptionskip=-1ex,
% breakautoindent = true,
% breakindent = 2em,
% breaklines = true,
 emptylines=10
}

\lstdefinestyle{abs} {
style=absnobg,
framerule=1pt,
backgroundcolor=\color{blue!5},
rulecolor=\color{gray!50},
frame=tblr
}

% make abs the default style
%\lstset{
%style=abs
%}

\lstdefinestyle{absexample}{
style=abs,
%backgroundcolor=\color{green!80!black!10},
}

\newcommand{\absinline}[1]{\lstinline[style=abs]{#1}}
\newcommand{\absinlinesmall}[1]{\lstinline[style=abs,basicstyle=\sourcefont\footnotesize\upshape,
 commentstyle=\commentfont\footnotesize,
 keywordstyle=\bfseries\color{keywordcolor}\footnotesize\upshape]{#1}}

\lstnewenvironment{abscode}{
\lstset{style=abs}}
{}

\lstnewenvironment{abscodesmall}{
\lstset{style=abs, basicstyle=\sourcefont\footnotesize\upshape,
 commentstyle=\commentfont\footnotesize,
 keywordstyle=\bfseries\color{keywordcolor}\footnotesize\upshape
}}
{}

\lstnewenvironment{abscodesmallnobg}{
\lstset{style=absnobg, basicstyle=\sourcefont\footnotesize\upshape,
 commentstyle=\commentfont\footnotesize,
 keywordstyle=\bfseries\color{keywordcolor}\footnotesize\upshape
}}
{}

\lstnewenvironment{absgrammar}{%
~\\[1ex]\noindent\textbf{Syntax:}
\lstset{style=absgrammar}}
{}



%%%%%%%%%%%%%%%%%%%%%%%%%%%%%%%%%%%%%%%%%%%%%%%%%%%%%%%%%%%%

\newcommand{\grsh}[1]{{\grshape #1}}

\newenvironment{grprode}[2]
{{\grshape #1}\,:
\\\indent\indent{\grshape #2}}
{\vspace{.75\baselineskip}}

\newcommand{\gprod}[2]{\begin{grprode}{#1}{#2}\end{grprode}}   
\newcommand{\gnewline}{\\\mbox{}\hspace{4em}}   


\newcommand{\gexnewline}{\\\mbox{}\indent}   

%\newcommand{\gnlbar}{\gnewline\mbox{}\hspace{-1em}\gbar}
%or
\newcommand{\gnlbar}{\gnewline\gbar}

\lstnewenvironment{absexample}
{\subsubsection*{Example:}
\lstset{style=absexample}}
{}

\newcommand{\kwdecl}{\upshape\ttfamily}
\newcommand{\mcode}[1]{{\texttt{#1}}} %mycode

\newcommand{\kwlbrace} {\kw{\{\,}}
\newcommand{\kwrbrace} {\kw{\}}}
\newcommand{\kwlparen} {\kw{(\,}}
\newcommand{\kwrparen} {\kw{)}}
\newcommand{\kwdot} {\kw{.}}

\newcommand{\kwassign} {$\,\kw{=}\,$}
\newcommand{\kwsemi} {$\,\kw{;}\,$}
\newcommand{\kwbar} {$\,\kw{|}\,$}
\newcommand{\kwcomma} {$\,\kw{,}\,$}
\newcommand{\kwlt} {$\,\kw{<}$}
\newcommand{\kwgt} {$\kw{>}\,$}
\newcommand{\kwclass} {\kw{class}}
\newcommand{\kwinterface} {\kw{interface}}
\newcommand{\kwextends} {\kw{extends}}
\newcommand{\kwdata} {\kw{data}}
\newcommand{\kwdef} {\kw{def}}
\newcommand{\kwimplements} {\kw{implements}}
\newcommand{\kwwhile} {\kw{while}}
\newcommand{\kwassert} {\kw{assert}}
\newcommand{\kwreturn} {\kw{return}}
\newcommand{\kwfut} {\kw{Fut}}
\newcommand{\kwskip} {\kw{skip}}
\newcommand{\kwget} {\kw{get}}
\newcommand{\kwnull} {\kw{null}}
\newcommand{\kwawait} {\kw{await}}
\newcommand{\kwif} {\kw{if}}
\newcommand{\kwthen} {\kw{then}}
\newcommand{\kwelse} {\kw{else}}
\newcommand{\kwsuspend} {\kw{suspend}}
\newcommand{\kwnew} {\kw{new}}
\newcommand{\kwthis} {\kw{this}}
\newcommand{\kwcase} {\kw{case}}
\newcommand{\kwlet} {\kw{let}}
\newcommand{\kwin} {\kw{in}}
\newcommand{\kwcog} {\kw{cog}}
\newcommand{\kwtype} {\kw{type}}
\newcommand{\kwguardand} {\kw{\&}}
\newcommand{\kwbang} {\kw{!}}


%%%%%%%%%%%%%%%%%%%%%%%%%%%%%%%%%%%%%%%%%%%%%%%%%%%%%%%%%%%%
%% tikz styles

\tikzstyle{box} = [draw=black,thick,bottom color=black!20,top color=white]
\tikzstyle{bx2} = [draw=black,thick,bottom color=blue!40,top color=white]
\tikzstyle{art} = [draw=black,rounded corners=7pt, thick,
  fill=blue!20, text=black, inner sep=2mm,font=\sffamily]
\tikzstyle{art2}= [art, bottom color=blue!10!gray!40, top color=white]


\pgfdeclarelayer{background}
\pgfdeclarelayer{cogs}
\pgfdeclarelayer{main}
\pgfsetlayers{background,cogs,main}


\newcommand{\diagramfont}{\sffamily}
\tikzstyle{lightshadow}=[drop shadow={shadow scale=1, fill=black!20,
%opacity=.5,
shadow xshift=1.2pt, shadow yshift=-1.2pt}]

\tikzstyle{ultralightshadow}=[
drop shadow={shadow scale=1, fill=black!10,
shadow xshift=2.5pt, shadow yshift=-2.5pt}]

\tikzstyle{diagramnodeWithoutShadow}=[draw=black!50,fill=white,font=\diagramfont,
line width=0.8pt, text depth=0ex]

\tikzstyle{diagramnode}=[diagramnodeWithoutShadow,
lightshadow]


\tikzstyle{object}=[diagramnode,rectangle,rounded corners=2pt,
inner sep=2mm,fill=black!2,fill=Lavender!80,%fill=blue!20!white,
%shadow xshift=0.2ex, shadow yshift=-0.2ex}
%path fading=south, shading angle=45}
]
\tikzstyle{mainobject}=[object,font=\diagramfont\bfseries]
\tikzstyle{cogform}=[rectangle,line width=1.5pt,draw=black!20,
text depth=0ex,inner sep=6pt,rounded corners]
\tikzstyle{cog}=[cogform, ultralightshadow, fill=black!2,
%draw=blue!50,fill=blue!10,
] % inner sep = 0.45cm

\tikzstyle{legendtext}=[right,text depth=0,font=\diagramfont\footnotesize]
\tikzstyle{legendsymbol}=[left,text depth=0,font=\diagramfont\footnotesize]

\tikzstyle{myarrow}=[single arrow,draw=gray!25,fill=gray!10,very thick]


%%% Local Variables: 
%%% mode: latex
%%% TeX-master: "main"
%%% End: 
