%!TEX root=main.tex





\section{Functions}
\emph{Functions} in ABS define names for parametrized data expressions.
A Function in ABS is always side effect-free, which means that it cannot
manipulate the Heap.

\begin{abssyntax}
\NT{FunctionDecl} \defn \TR{def}\ \TR{Type}\ \NT{Identifier}\ \OPTG{\TRS{<}\ \NT{TypeIdList}\ \TRS{>}}\ \TRS{(} \NT{ParamList} \TRS{)}\ \TRS{=}\ 
                 \NT{FunBody}\ \TRS{;}  \\
\NT{FunBody}      \defn \TR{builtin} ~|~ \NT{PureExp}\\
\NT{TypeIdList}   \defn \NT{TypeId}\ \MANYG{\TRS{,}\ \NT{TypeId}}
\end{abssyntax}

\begin{absexample}
def Int length(IntList list) =
  case list { 
    Nil => 0;
    Cons(n, ls) => 1 + length(ls);
  };
\end{absexample}

\subsection{Parametric Functions}
\label{sec:parametric-functions}

\emph{Parametric Functions} serve to work with parametric data types in a
general way.  For example, given a list of any type, a parametric
function \absinline{head} can return the first element, regardless of its
type.  Parametric functions are defined like normal functions but have
an additional \texttt{type parameter} section inside broken brackets
(\texttt{< >}) after the function name.

\begin{absexample}
def A head<A>(List<A> list) =
  case list {
    Cons(x, xs) => x;
  };
\end{absexample}




\section{Pure Expressions}
\emph{Pure Expressions} are side effect-free expressions. This means that these expressions cannot modify the Heap.

\begin{abssyntax}
\NT{PureExp}     \defn \NT{Variable}
              \defc  \NT{FieldAccess}
              \defc  \NT{ThisExp}
              \defc  \NT{NullLiteral}
              \defc  \NT{LetExp}
              \defc  \NT{DataConstrExp}
              \defc  \NT{FnAppExp}
              \defc  \NT{FnAppListExp}
              \defc  \NT{CaseExp}
              \defc  \NT{OperatorExp}
              \defc  \TRS{(} \NT{PureExp} \TRS{)}\\
\NT{Variable}    \defn \NT{Identifier}\\
\NT{FieldAccess} \defn \NT{ThisExp}\ \TRS{.}\ \NT{Identifier}\\
\NT{ThisExp}     \defn \TR{this}\\
\NT{PureExpList} \defn \NT{PureExp}\ \MANYG{\TRS{,}\ \NT{PureExp}}
\end{abssyntax}


\subsection{Let Expressions}
\emph{Let Expressions} bind variable names to pure expressions.

\begin{abssyntax}
\NT{LetExp} \defn \TR{let}\ \TRS{(} \NT{Param} \TRS{)}\ \TRS{=}\ \NT{PureExp}\ \TR{in}\ \NT{PureExp}
\end{abssyntax}

\begin{absexample}
let (Bool x) = True in !x  
\end{absexample}

\subsection{Data Type Constructor Expressions}
\emph{Data Type Constructor Expressions} are expressions that create data type values by using data type constructors.
Note that for data type constructors that have no parameters the parentheses are optional.

\begin{abssyntax}
\NT{DataConstrExp} \defn \NT{TypeName}
                   \defc \NT{TypeName}\ \TRS{(} \OPT{\NT{PureExpList}} \TRS{)}
\end{abssyntax}

\begin{absexample}
True
Cons(True, Nil)  
ABS.StdLib.Nil
\end{absexample}

\subsection{Function Applications}
\emph{Function Applications} apply functions to arguments.

\begin{abssyntax}
\NT{FnAppExp} \defn \NT{Name}\ \TRS{(} \OPT{\NT{PureExpList}} \TRS{)}   
\end{abssyntax}

\begin{absexample}
tail(Cons(True, Nil))
ABS.StdLib.head(list)
\end{absexample}

\subsection{Case Expressions / Pattern Matching}
ABS supports pattern matching by the \emph{Case Expression}.
It takes an expression as first argument, which a series of patterns is matched against.
When a pattern matches, the corresponding expression on the right hand side is evaluated.

\begin{abssyntax}
\NT{CaseExp}       \defn \TR{case}\ \NT{PureExp}\ \TRS{\{} \MANY{\NT{CaseBranch}}\ \TRS{\}}\\
\NT{CaseBranch}    \defn \NT{Pattern}\ \TRS{=>}\ \NT{PureExp}\ \TRS{;}\\
\NT{Pattern}       \defn \NT{Identifier}
                \defc  \NT{Literal}
                \defc  \NT{ConstrPattern}
                \defc  \TRS{\_}\\
\NT{ConstrPattern} \defn \NT{TypeName}\ \OPTG{\TRS{(}\ \OPT{\NT{PatternList}}\ \TRS{)}}\\
\NT{PatternList}   \defn \NT{Pattern}\ \MANYG{\TRS{,}\ \NT{Pattern}}
\end{abssyntax}

% \begin{abscode}
% case <expr> {
%   <pattern> => <expr> ;
%   <pattern> => <expr> ;
%   ...
% }  
% \end{abscode}

\subsubsection{Patterns}
There are five different kinds of patterns available in ABS:
\begin{itemize}
\item Pattern Variables (e.g., \absinline{x}, where \absinline{x} is not bound yet)
\item Bound Variables (e.g., \absinline{x}, where \absinline{x} is bound)
\item Literal Patterns (e.g., \absinline{5})
\item Data Constructor Patterns (e.g.,~\absinline{Cons(Nil,x)})
\item Underscore Pattern (\absinline{_})
\end{itemize}

\paragraph{Pattern Variables}
Pattern variables are simply unbound variables.
Like the underscore pattern, these variables match every value, but, in addition, bind the variable to the matched value. The bound variable can then be used in the right-hand-side expression of the corresponding branch.
Typically, pattern variables are used inside of data constructor patterns to extract values from data constructors.
For example:
\begin{abscode}
def A fromJust<A>(Maybe<A> a) = 
  case a { 
    Just(x) => x; 
  };
\end{abscode}

\paragraph{Bound Variables}
If a bound variable is used as a pattern, the pattern matches if the value of the case expression is equal to the value of the bound variable.

\begin{abscode}
def Bool contains<A>(List<A> list, A value) =
  case list {
    Nil => False;
    Cons(value, _) => True;
    Cons(_, rest) => contains(rest, value);
  };
\end{abscode}

\paragraph{Literal Patterns}
Literals can be used as patterns. This is similar to bound variables, because the pattern matches if the value of the case expression is equal to the literal value.

\begin{abscode}
def Bool isEmpty(String s) =
  case b {
    "" => True;
    _  => False;
  };
\end{abscode}

\paragraph{Data Constructor Patterns}
A data constructor pattern is like a standard data constructor expression, but where certain sub expressions can be patterns again.

\begin{abscode}
def Bool negate(Bool b) =
  case b {
    True => False;
    False => True;
  };
\end{abscode}

\begin{abscode}
def List<A> remainder(List<A> list) =
  case b {
    Cons(_, rest) => rest;
  };
\end{abscode}

\paragraph{Underscore Pattern.}
The underscore pattern (\verb!_!) simply matches every value. 
It is in general used as the last pattern in a case expression to define a default case.
For example:
\begin{abscode}
def Bool isNil<A>(List<A> list) =
  case list {
    Nil => True;
    _ => False;
  };
\end{abscode}


% \subsubsection{Error Cases}
% It is currently backend dependent what happens if there is a case expression which does not match any pattern.
% The Java backend throws an \verb_UnmatchedCaseException_ exception in this case.

\subsubsection{Type Checking}
A case expression is type-correct if and only if all its expressions and all its branches are type-correct and the right-hand side of all branches have a common super type. This common super type is also the type of the overall case expression. 

A branch (a pattern and its expression) is type-correct if its pattern and its right-hand side expression are type-correct.
A pattern is type-correct if it can match the corresponding case expression.

\subsection{Operator Expressions}
ABS has a number of predefined operators which can be used to form \emph{Operator Expressions}.

\begin{abssyntax}
\NT{OperatorExp} \defn \NT{UnaryExp}
              ~|~ \NT{BinaryExp}\\
\NT{UnaryExp}    \defn \NT{UnaryOp}\ \NT{PureExp}\\
\NT{UnaryOp}     \defn \verb*_ _ ~|~ \TRS{-}\\
\NT{BinaryExp}   \defn \NT{PureExp}\ \NT{BinaryOp}\ \NT{PureExp}\\
\NT{BinaryOp}    \defn \TRS{==} ~|~ \TRS{!=} ~|~ \TRS{<} ~|~ \TRS{<=} ~|~ \TRS{>} ~|~ \TRS{>=} ~|~ \TRS{+} ~|~ \TRS{-} ~|~ \TRS{*} ~|~ \TRS{/} ~|~ \TRS{\%}
\end{abssyntax}

\begin{table}[ht]
\centering
 \renewcommand{\arraystretch}{1.5} 
 \begin{tabular}[h]{| l | l | l | l | l |}
   \hline
   Expression & Meaning & Associativity & Argument types & Result type \\
   \hline
\mcode{e1 || e2}  & logical or & left & Bool, Bool & Bool \\ \hline 
\mcode{e1 \&\& e2}  & logical and & left & Bool, Bool & Bool \\ \hline
\mcode{e1 == e2}& equality & left & compatible & Bool \\ 
\mcode{e1 != e2}& inequality & left & compatible & Bool \\ \hline 
\mcode{e1 < e2}& less than & left & Int, Int & Bool \\
\mcode{e1 <= e2}& less than or equal to  & left & Int, Int & Bool \\
\mcode{e1 > e2}& greater than & left & Int, Int & Bool \\
\mcode{e1 >= e2}& greater than or equal to  & left & Int, Int & Bool \\ \hline
\mcode{e1 + e2}& concatenation & left & String, String & String \\
\mcode{e1 + e2}& addition & left & Int, Int & Int \\
\mcode{e1 - e2}& subtraction & left & Int, Int & Int \\\hline
\mcode{e1 * e2}& multiplication & left & Int, Int & Int \\
\mcode{e1 / e2}& division & left & Int, Int & Int \\
\mcode{e1 \% e2}& modulo & left & Int, Int & Int \\ \hline 
\verb_! e_  & logical negation  & right & Bool  & Bool \\ 
\verb_- e_   & integer negation  & right & Int  & Int \\ \hline 
 \end{tabular}
  \caption{\label{fig:operatorExpressions} Operator expressions, grouped according to precedence from low to high.}
\end{table} 
Table~\ref{fig:operatorExpressions} describes the meaning as well as the associativity and the precedence of the different operators.
They are grouped according to precedence, as indicated by horizontal
rules, from low precedence to high precedence.

\section{Expression With Side Effects}
Beside pure expressions, ABS has expressions with side effects.
However, these expressions are defined in such a way that they can only have a single side effect. This means that subexpressions of expressions can only be pure expressions again. This restriction simplifies the reasoning about ABS expressions.
\newpage

\begin{abssyntax}
\NT{Exp}    \defn \NT{PureExp}
         ~|~ \NT{EffExp}\\
\NT{EffExp} \defn \NT{NewExp}
         \defc  \NT{SyncCall}
         \defc  \NT{AsyncCall}
         \defc  \NT{GetExp}
\end{abssyntax}

\subsection{New Expression}
A \emph{New Expression} creates a new object from a class name and a list of arguments. In ABS objects can be created in two different ways.
Either they are created in the current COG, using the \absinline{new local} expression,
or they are created in a new COG by using the \absinline{new} expression.

\begin{abssyntax}
\NT{NewExp} \defn \TR{new}\ \OPT{\TR{local}}\ \NT{TypeName}\ \TRS{(} \NT{PureExpList} \TRS{)}  
\end{abssyntax}

\begin{absexample}
new local Foo(5)
new Bar()
\end{absexample}

\subsubsection{Standard Object Creation}
When using the \absinline{new local} expression, the new object is created in the \emph{current} COG, i.e., the COG of the current receiver object.
Figure~\ref{fig:newExpr} illustrates this by showing two different runtime states, one before the creation of an object \verb_b_ and one after its creation.

\begin{figure}
\centering
\begin{tikzpicture}
\node[object] (a) {this:A};
\begin{pgfonlayer}{cogs}
  \node[cog, fit=(a)] (coga) {};
\end{pgfonlayer}
\node[myarrow, right=2cm] (arrow) at (coga) {\verb_new local B()_};

\node[object, right=1cm] (a2) at (arrow.east) {this:A};
\node[object, right=1cm] (b2) at (a2) {b:B};
\begin{pgfonlayer}{cogs}
  \node[cog, fit=(a2) (b2)] (cogb) {};
\end{pgfonlayer}

\node[fit=(coga) (arrow) (cogb)] (pic) {};
\matrix[below=0.5cm] (legend) at (pic.south) {
   \node[object,legendsymbol] {~}; & \node[legendtext] {objects}; 
 & \node[cog,legendsymbol] {~}; & \node[legendtext] {COGs}; 
\\
};
\draw[gray!50] (legend.north west) -- (legend.north east);
\end{tikzpicture}
\caption{Process of creating an object inside the current COG by using the \texttt{new local} expression.}
\label{fig:newExpr}
\end{figure}

\subsubsection{COG Object Creation}
The concurrency model of ABS is based on the notion of COGs~\cite{johnsen10fmco}.
An ABS system at runtime is a set of concurrently running COGs. A COGs can be seen as an isolated subsystem, which has its own state (an object-heap) and its own internal behavior.
COGs are created implicitly when creating a new object by using the \verb_new_ expression.
Figure~\ref{fig:newCogExpr} illustrates this by showing two different runtime states, one before the creation of an object \verb_b_ using the \verb_new cog_ expression and one after its creation. In the second runtime state, two COGs exists. 

\begin{figure}
\centering
\begin{tikzpicture}
\node[object] (a) {this:A};
\begin{pgfonlayer}{cogs}
  \node[cog, fit=(a)] (coga) {};
\end{pgfonlayer}
\node[myarrow, right=2cm] (arrow) at (coga) {\verb_new B()_};

\node[object, right=1cm] (a2) at (arrow.east) {this:A};
\node[object, right=1cm] (b2) at (a2.east) {b:B};

\begin{pgfonlayer}{cogs}
  \node[cog, fit=(a2)] (coga2) {};
  \node[cog, fit=(b2)] (cogb2) {};
\end{pgfonlayer}



\node[fit=(coga) (arrow) (cogb2)] (pic) {};
\matrix[below=0.5cm] (legend) at (pic.south) {
   \node[object,legendsymbol] {~}; & \node[legendtext] {objects}; 
 & \node[cog,legendsymbol] {~}; & \node[legendtext] {COGs}; 
\\
};
\draw[gray!50] (legend.north west) -- (legend.north east);
\end{tikzpicture}
\caption{Process of creating an object in a new COG by using the \texttt{new} expression.}
\label{fig:newCogExpr}
\end{figure}


\subsection{Synchronous Call Expression}
A \emph{Synchronous Call} consists of a target expression, a method name, and a list of argument expressions.

\begin{abssyntax}
\NT{SyncCall} \defn \NT{PureExp}\ \TRS{.}\ \NT{Identifier}\ \TRS{(} \NT{PureExpList} \TRS{)}  
\end{abssyntax}

\begin{absexample}
Bool b = x.m(5);  
\end{absexample}

\subsection{Asynchronous Call Expression}
An \emph{Asynchronous Call} consists of a target expression, a method name, and a list of argument expressions.
Instead of directly invoking the method, an asynchronous method call creates a new \emph{task} in the target COG, which is executed asynchronously. This means that the calling task proceeds independently after the call, without waiting for the result~\cite{johnsen10fmco}.
The result of an asynchronous method call is a future (\verb_Fut<V>_), which can be used by the calling
task to later obtain the result of the method call.
That future is \emph{resolved} by the task that has been created in the target COG to execute the method.

\begin{abssyntax}
\NT{AsyncCall} \defn \NT{PureExp}\ \TRS{!}\ \NT{Identifier}\ \TRS{(} \NT{PureExpList} \TRS{)}  
\end{abssyntax}

\begin{absexample}
Fut<Bool> f = x!m(5);  
\end{absexample}

\subsection{Get Expression}\label{sec:getexpr}
A \emph{Get Expression} is used to obtain the value from a future.
The current task is blocked until the value of the future is available, i.e., until
the future has been resolved. No other task in the COG can be activated in
the meantime~\cite{johnsen10fmco}.

\begin{abssyntax}
\NT{GetExp} \defn \NT{PureExp}\ \TRS{.}\ \TR{get}
\end{abssyntax}

\begin{absexample}
Bool b = f.get;
\end{absexample}


\section{Statements}
In contrast to expressions, \emph{Statements} in ABS are not
evaluated to a value. If one wants to assign a value to statements
it would be the \absinline{Unit} value.

\begin{abssyntax}
\NT{Statement}    \defn \NT{CompoundStmt}
               \defc  \NT{VarDeclStmt}
               \defc  \NT{AssignStmt}
               \defc  \NT{AwaitStmt}
               \defc  \NT{SuspendStmt}
               \defc  \NT{SkipStmt}
               \defc  \NT{AssertStmt}
               \defc  \NT{ReturnStmt}
               \defc  \NT{ExpStmt}\\
\NT{CompoundStmt} \defn \NT{Block}
               \defc  \NT{IfStmt}
               \defc  \NT{WhileStmt}
\end{abssyntax}

\subsection{Block}
A block consists of a sequence of statements and defines a name scope for variables.

% \subsubsection{Type Checking}
% A block is type-correct if all its statements are type correct.


\begin{abssyntax}
\NT{Block} \defn \TRS{\{}\ \MANY{\NT{Statement}}\ \TRS{\}}
\end{abssyntax}

\subsection{If Statement}

\begin{abssyntax}
\NT{IfStmt} \defn \TR{if}\ \TRS{(} \NT{PureExp} \TRS{)}\ \NT{Stmt} \OPTG{\TR{else}\ \NT{Stmt}}
\end{abssyntax}

\begin{absexample}
if (5 < x) {
  y = 6;
} else {
  y = 7;
}

if (True)
  x = 5;
\end{absexample}

\subsection{While Statement}

\begin{abssyntax}
\TR{while}\ \TRS{(} \NT{PureExp} \TRS{)}\ \NT{Stmt}
\end{abssyntax}

\begin{absexample}
while (x < 5)
   x = x + 1;
\end{absexample}

\subsection{Variable Declaration Statements}
A variable declaration statement is used to declare variables.

\begin{abssyntax}
\NT{VarDeclStmt} \defn \NT{TypeName}\ \NT{Identifier}\ \OPTG{\TRS{=}\ \NT{Exp}}\ \TRS{;}  
\end{abssyntax}

A variable has an optional \emph{initialization expression} for defining the initial value of the variable. The initialization expression is \emph{mandatory} for variables of data types.
It can be left out only for variables of reference types, in which case the variable is initialized with \absinline{null}.

\begin{absexample}
Bool b = True;  
\end{absexample}

\subsection{Assign Statement}
The \emph{Assign Statement} assigns a value to a variable or a field.

\begin{abssyntax}
\NT{AssignStmt} \defn \NT{Variable}\ \TRS{=}\ \NT{PureExp}\ \TRS{;}
             \defc  \NT{FieldAccess}\ \TRS{=}\ \NT{PureExp}\ \TRS{;}
\end{abssyntax}

\begin{absexample}
this.f = True;
x = 5;
\end{absexample}


\subsection{Await Statement}
\emph{Await Statements} suspend the current task until the given \emph{guard} is true~\cite{johnsen10fmco}. 
While the task is suspended, other tasks within the same COG can be activated. 
Await statements are also called \emph{scheduling points}, because they are the only source positions,
where a task may become suspended and other tasks of the same COG can be activated.


\begin{abssyntax}
\NT{AwaitStmt}  \defn \TR{await}\ \NT{Guard}\ \TRS{;}\\
\NT{Guard}      \defn \NT{ClaimGuard}
             \defc  \NT{PureExp} 
             \defc  \NT{Guard}\ \TRS{\&}\ \NT{Guard}\\
\NT{ClaimGuard} \defn \NT{Variable}\ \TRS{?}
             \defc  \NT{FieldAccess}\ \TRS{?}
\end{abssyntax}

\begin{absexample}
Fut<Bool> f = x!m();
await f?;
await this.x == True;
await f? & this.y > 5;
\end{absexample}

\subsection{Suspend Statement}
A \emph{Suspend Statement} leads to a scheduling point, i.e. the currently active task releases the processor, allowing other tasks within the same COG to become active.

\begin{abssyntax}
\NT{SuspendStmt} \defn \TR{suspend}\ \TRS{;}
\end{abssyntax}

\begin{absexample}
suspend;
\end{absexample}

\subsection{Skip Statement}
The \emph{Skip Statement} is a statement that does nothing.

\begin{abssyntax}
\NT{SkipStmt} \defn \TR{skip}\ \TRS{;}
\end{abssyntax}

\subsection{Assert Statement}\label{sec:abs:assert}
An \emph{Assert Statement} is a statement for asserting certain conditions.
\newpage

\begin{abssyntax}
\NT{AssertStmt} \defn \TR{assert}\ \NT{PureExp}\ \TRS{;}
\end{abssyntax}

\begin{absexample}
assert x != null;
\end{absexample}

\subsection{Return Statement}
A \emph{Return Statement} defines the return value of a method.
A return statement can only appear as a last statement in a method body.

\begin{abssyntax}
\NT{ReturnStmt} \defn \TR{return}\ \NT{Exp}\ \TRS{;}
\end{abssyntax}

\begin{absexample}
return x;
\end{absexample}

\subsection{Expression Statement}
An \emph{Expression Statement} is a statement that only consists of a single expression. Such statements are only executed for the effect of the expression. 

\begin{abssyntax}
\NT{ExpStmt} \defn \NT{Exp}\ \TRS{;}
\end{abssyntax}

\begin{absexample}
new local C(x);
\end{absexample}

\section{Classes and Interfaces}
\label{sec:classandint}
Objects in ABS are built from \emph{classes}, which implement \emph{interfaces}. Only interfaces can be used as types in ABS.

\subsection{Interfaces}
\label{sec:interfacedecl}
Interfaces in ABS are similar to interfaces in Java.
They have a name, which defines a nominal type, and 
they can \emph{extend} arbitrary many other interfaces.
The interface body consists of a list of method signature declarations. 
Method names start with a lowercase letter.

\begin{abssyntax}
\NT{InterfaceDecl} \defn \TR{interface}\ \NT{TypeId}\ \OPTG{\TR{extends}\
  \NT{TypeName}\ \MANYG{\TRS{,}\ \NT{TypeName}}}\
  \TRS{\{}\ \MANY{\NT{MethSig}}\ \TRS{\}}\\
%\NT{MethSigList}   \defn \MANY{\NT{MethSig}}\\
\NT{MethSig}       \defn \NT{Type}\ \NT{Identifier}\ \TRS{(}\ \OPT{\NT{ParamList}}\ \TRS{)}\ \TRS{;}\\
\NT{ParamList}     \defn \NT{Param}\ \MANYG{\TRS{,}\ \NT{Param}}\\
\NT{Param}         \defn \NT{Type}\ \NT{Identifier}
\end{abssyntax}

%\begin{absexample}
%interface Foo {
%   Unit m(Bool b, Int i);
%}
%
%interface Bar extends Foo {
%   Bool n();    
%}
%\end{absexample}

The interfaces in the example below represent a database system, providing
functionality to store and retrieve files, and a node of a peer-to-peer file
sharing system. Each node of a peer-to-peer system plays both the role of a
server and a client. The data types are defined in the ABS standard library,
included in Appendix~\ref{ch:absstdlib}, and the remainder types are type synonyms
included in Section~\ref{sec:typesynonyms}.

\begin{absexample}
interface DB {
  File getFile(Filename fId);
  Int getLength(Filename fId);
  Unit storeFile(Filename fId, File file);
  Filenames listFiles();
}

interface Client {
  List<Pair<Server,Filenames>> availFiles(List<Server> sList);
  Unit reqFile(Server sId, Filename fId);
}

interface Server {
  Filenames inquire();
  Int getLength(Filename fId);
  Packet getPack(Filename fId, Int pNbr);
}

interface Peer extends Client, Server {
  List<Server> getNeighbors();
}
\end{absexample}

\subsection{Classes}
Like in typical class-based languages, classes in ABS 
are used to create objects. 
Classes can implement an arbitrary number of interfaces.
ABS does not support inheritance, as code reuse in ABS is realized by delta modules.
Classes do not have constructors in ABS but instead have
\emph{class parameters} and an optional \emph{init block}.
Class parameters actually define additional fields of the 
class that can be used like any other declared field.
\newpage

\begin{abssyntax}
\NT{ClassDecl}     \defn \TR{\NT{class}}\ \NT{TypeId}\ \OPTG{\TRS{(} \NT{ParamList} \TRS{)}}\ \OPTG{\TR{implements}\ \NT{TypeName}\ \MANYG{\TRS{,}\ \NT{TypeName}}} 
\\&&
                  \TRS{\{}\ \OPT{\NT{FieldDeclList}}\ \OPT{\NT{Block}}\ \OPT{\NT{MethDeclList}}\ \TRS{\}}\\  
\NT{FieldDeclList} \defn \NT{FieldDecl}\ \MANYG{\TRS{,}\ \NT{FieldDecl}}\\
\NT{FieldDecl}     \defn \NT{TypeId}\ \NT{Identifier}\ \OPTG{\TRS{=}\ \NT{PureExp}}\ \TRS{;}\\
\NT{MethDeclList}  \defn \NT{MethDecl}\ \MANYG{\TRS{,}\ \NT{MethDecl}}\\
\NT{MethDecl}      \defn \NT{Type}\ \NT{Identifier}\ \TRS{(} \NT{ParamList} \TRS{)}\ \NT{Block}
\end{abssyntax}

%\begin{absexample}
%class Foo(Bool b, Int i) implements Bar, Baz {
%   Int j = 5;
%   Bar b;
%
%   {   
%     j = i;
%   }  
%
%   Bool m() { 
%     return True;
%   }
%}
%\end{absexample}

We continue peer-to-peer example with an implementation of the \absinline{DB}
interface, and the signature of a node that implements the \absinline{Peer}
interface. The full implementation of a node is presented in Section~\ref{app:example}.
%The remainder types, 
%\absinline{Filename}, \absinline{Filenames}, \absinline{File}, \absinline{Packet}, and \absinline{Catalog}, are defined as type synonyms, that we will introduce 
%
%... sec:typesynonyms

\begin{absexample}
class DataBase(Map<Filename,File> db) implements DB {
  File getFile(Filename fId) {
      return lookup(db, fId);
  }
  
  Int getLength(Filename fId){
      return length(lookup(db, fId));
  }
  
  Unit storeFile(Filename fId, File file) {
      db = insert(Pair(fId,file), db);
  }
  
  Filenames listFiles() {
      return keys(db);
  }
}

class Node(DB db, Peer admin, Filename file) implements Peer {
  Catalog catalog;
  List<Server> myNeighbors;

  // implementation...
}
\end{absexample}

\subsubsection{Active Classes}
\label{sec:active-classes}

A class can be \emph{active} or \emph{passive}.  Active classes start an
activity ``on their own upon creation, passive classes only react to
incoming method calls.

A class is active if and only if it has a \emph{run method}:
\begin{absexample}
Unit run() {
  // active behavior ...
}
\end{absexample}
The run method is called after object initialization.  

\section{Annotations}
ABS supports \emph{Annotations} to enrich an ABS model with additional information, for example, to realize pluggable type systems.
Annotations can appear before any declaration and type usage in ABS programs (which is not given in the grammar definitions, to improve readability).

\begin{abssyntax}
\NT{Annotation} \defn \TRS{[}\ \OPTG{\NT{TypeName}\ \TRS{:}}\ \NT{PureExp}\ \TRS{]}
\end{abssyntax}

\begin{absexample}
[LocationType:Near] Peer p;
[Far] Network n;
List<[Near] Peer> peers = Nil;
\end{absexample}

\subsection{Type Annotations}
ABS has a predefined ``meta-annotation \absinline{TypeAnnotation} to declare annotations to be \emph{Type Annotations}.
Data types that are annotated with that annotation are specially treated by the ABS compiler to support an easier implementation of pluggable type systems.

\begin{absexample}
[TypeAnnotation]
data LocationType = Far | Near | Somewhere | Infer;
\end{absexample}


\section{Modules}\label{sec:modules}
For name spacing, code structuring, and code hiding purposes, ABS offers a module system.
The module system of ABS is very similar to that of Haskell \cite{Haskell98}.
It uses, however, a different syntax that is similar to that of Java \cite{gosling96} and Python.
\newpage

\begin{abssyntax}
\NT{ModuleDecl}  \defn \TR{module}\ \NT{TypeName}\ \TRS{;}\ \OPT{\NT{ExportList}}\ \OPT{\NT{ImportList}}\ \MANY{\NT{Decl}}\ \OPT{\NT{Block}}\\
\NT{ExportList}  \defn \NT{Export}\ \MANYG{\TRS{,}\ \NT{Export}}\\
\NT{ImportList}  \defn \NT{Import}\ \MANYG{\TRS{,}\ \NT{Import}}\\
\NT{Export}      \defn \TR{export}\ \NT{AnyNameList}\ \OPTG{\TR{from}\ \NT{TypeName}}\ \TRS{;}
            \defc \TR{export}\ \TRS{*}\ \OPTG{\TR{from}\ \NT{TypeName}}\ \TRS{;}\\
\NT{Import}      \defn \TR{import}\ \NT{AnyNameList}\ \OPTG{\TR{from}\ \NT{TypeName}}\ \TRS{;}
            \defc \TR{import}\ \TRS{*}\ \TR{from}\ \NT{TypeName}\ \TRS{;}\\
\NT{AnyNameList} \defn \NT{AnyName}\ \OPTG{\TRS{,}\ \NT{AnyName}}\\
\NT{AnyName}     \defn \NT{Name} ~|~ \NT{TypeName}\\
\NT{Decl}        \defn \NT{FunDecl} ~|~ \NT{TypeSynDecl} ~|~ \NT{DataTypeDecl}
                 \defc
                \NT{InterfaceDecl} ~|~ \NT{ClassDecl} ~|~ \NT{DeltaDecl}
\end{abssyntax}


A module with name \absinline{MyModule} is declared by writing 
\begin{abscode}
module MyModule;
\end{abscode}
This declaration introduces a new module name \absinline{MyModule} which can be
used to qualify names.
All declarations which follow this statement belong to the module \absinline{MyModule}.
A module name is a type name and must always start with an upper case letter.

\subsection{Exporting}
By default, modules do not export any names.
In order to make names of a module usable to other modules, the names have to be \emph{exported}.
Exporting is done by writing one or several \emph{exports} after the module declaration.
% \begin{abscode}
% export <Name>, <Name>, ... , <Name> ;
% export <Name>, ...
% \end{abscode}
%Where the exported \verb_<Name>_ refers to any name that is visible in the %scope of the module.
For example, to export a data type and a data constructor, one can write something like this:
\begin{abscode}
module Drinks;
export Drink, Milk;
data Drink = Milk | Water;
\end{abscode}

Note that in this example, the data constructor \absinline{Water} is not exported, and can thus not be used by other modules.
By only exporting the data type without any of its constructors one can realize \emph{abstract data types}.

\subsubsection{Exporting Everything}
Sometimes one wants to export everything from a module.
In that case one can write:
\begin{abscode}
export *;
\end{abscode}
In this case, all names that are \emph{defined} in the module are exported, in particular, this means that imported names are \emph{not} exported.

\subsection{Importing}
In order to use exported names of a module in another module, the names have to be \emph{imported}.
After the list of export statements follows an optional list of \emph{imports}, which are used to import names from other modules.
For example, to write a module that imports the \absinline{Drink} data type of the module \absinline{Drinks} one can write something like this:
\begin{abscode}
module Bar;
import Drinks.Drink;  
\end{abscode}
After a name has been imported, it can be used inside the module in a fully qualified way.

\subsubsection{Unqualified Importing}
To use a name from another module in an unqualified way one has to use \emph{unqualified imports}.
For example, to use the \absinline{Milk} data constructor inside the \absinline{Bar} module, without having to qualify it with the \absinline{Drinks} module each time, one uses the unqualified import statement:
\begin{abscode}
module Bar;
import Drinks.Drink;  
import Milk from Drinks;
\end{abscode}
Note that this kind of import also imports the qualified names. 
So in this example the names \absinline{Milk} and \absinline{Drinks.Milk} can be used inside the module \absinline{Bar}.

To use all exported names from another module in an unqualified way one can write:
\begin{abscode}
import * from SomeModule;
\end{abscode}

\subsection{Exporting Imported Names}
It is possible to export names that have been imported. For example,
\begin{abscode}
module Bar;
export Drink;
import * from Drinks;
\end{abscode}
exports data type \absinline{Drink} that has been imported from \absinline{Drinks}

To export all names imported from a certain module one can write
\begin{abscode}
export * from SomeModule;
\end{abscode}
In this case, all names that have been imported from module \absinline{SomeModule} are exported. For example,
\begin{abscode}
module Bar;
export * from Drinks;
import * from Drinks;
\end{abscode}
exports all names that are exported by module \absinline{Drinks}.

However, in this example:
\begin{abscode}
module Bar;
export * from Drinks;
import Drink from Drinks;
\end{abscode}
only \absinline{Drink} is exported as this is the only name imported from module \absinline{Drinks}.
Note: only names that are visible in a module can be exported by that module.

To only export some names from a certain module one can write, for example:
\begin{abscode}
module Bar;
export Drink from Drinks;
import * from Drinks;
\end{abscode}
only exports \absinline{Drink} from module \absinline{Drinks}

\section{Model}
\label{sec:model}
A \emph{Model} in ABS represents a type-closed set of \emph{Modules}.
A Module defines a set of declarations and an optional \emph{Main Block}.
Modules reside in \emph{Compilation Units}, which are typically represented by files ending with \texttt{.abs}.
A Model is thus set of Compilation Units.

\begin{abssyntax}
\NT{Model}           \defn \MANY{\NT{CompilationUnit}}\\
\NT{CompilationUnit} \defn \MANY{\NT{ModuleDecl}}
\end{abssyntax}






%\lstDeleteShortInline|

%%% Local Variables:
%%% mode: latex
%%% TeX-master: "absrefmanual"
%%% End:



