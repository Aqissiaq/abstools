\chapter{Statements}
In contrast to expressions, \emph{Statements} in ABS are not
evaluated to a value. If one wants to assign a value to statements
it would be the \absinline{Unit} value.

\begin{abssyntax}
\NT{Statement}    \defn \NT{CompoundStmt}
               \defc  \NT{VarDeclStmt}
               \defc  \NT{AssignStmt}
               \defc  \NT{CaseStmt}
               \defc  \NT{AwaitStmt}
               \defc  \NT{SuspendStmt}
               \defc  \NT{SkipStmt}
               \defc  \NT{AssertStmt}
               \defc  \NT{ReturnStmt}
               \defc  \NT{ExpStmt}\\
\NT{CompoundStmt} \defn \NT{Block}
               \defc  \NT{IfStmt}
               \defc  \NT{WhileStmt}
               \defc  \NT{TryCatchFinallyStmt}
\end{abssyntax}

\section{Block}
A block consists of a sequence of statements and defines a name scope for variables.

% \subsubsection{Type Checking}
% A block is type-correct if all its statements are type correct.


\begin{abssyntax}
\NT{Block} \defn \TRS{\{}\ \MANY{\NT{Statement}}\ \TRS{\}}
\end{abssyntax}

\section{If Statement}

\begin{abssyntax}
\NT{IfStmt} \defn \TR{if}\ \TRS{(} \NT{PureExp} \TRS{)}\ \NT{Stmt} \OPTG{\TR{else}\ \NT{Stmt}}
\end{abssyntax}

\begin{absexample}
if (5 < x) {
  y = 6;
} else {
  y = 7;
}

if (True)
  x = 5;
\end{absexample}

\section{While Statement}

\begin{abssyntax}
\TR{while}\ \TRS{(} \NT{PureExp} \TRS{)}\ \NT{Stmt}
\end{abssyntax}

\begin{absexample}
while (x < 5)
   x = x + 1;
\end{absexample}

\section{Variable Declaration Statements}
A variable declaration statement is used to declare variables.

\begin{abssyntax}
\NT{VarDeclStmt} \defn \NT{TypeName}\ \NT{Identifier}\ \OPTG{\TRS{=}\ \NT{Exp}}\ \TRS{;}
\end{abssyntax}

A variable has an optional \emph{initialization expression} for defining the initial value of the variable. The initialization expression is \emph{mandatory} for variables of data types.
It can be left out only for variables of reference types, in which case the variable is initialized with \absinline{null}.

\begin{absexample}
Bool b = True;
\end{absexample}

\section{Assign Statement}
The \emph{Assign Statement} assigns a value to a variable or a field.

\begin{abssyntax}
\NT{AssignStmt} \defn \NT{Variable}\ \TRS{=}\ \NT{Exp}\ \TRS{;}
             \defc  \NT{FieldAccess}\ \TRS{=}\ \NT{Exp}\ \TRS{;}
\end{abssyntax}

\begin{absexample}
this.f = True;
x = 5;
\end{absexample}

\section{Case Statement}

The case statement, like the case expression, takes an expression as
first argument, which is matched against a series of patterns.  The
effect of executing the case statement is the execution of the statement
(which can be a block) of the first branch whose pattern matches the
expression.  It is an error if no pattern matches the expression.


\begin{abssyntax}
\NT{CaseStmt}       \defn \TR{case}\ \NT{PureExp}\ \TRS{\{} \MANY{\NT{CaseStmtBranch}}\ \TRS{\}}\\
\NT{CaseStmtBranch}    \defn \NT{Pattern}\ \TRS{=>}\ \NT{Stmt}\ \TRS{;}\\
\NT{Pattern}       \defn \NT{Identifier}
                \defc  \NT{Literal}
                \defc  \NT{ConstrPattern}
                \defc  \TRS{\_}\\
\NT{ConstrPattern} \defn \NT{TypeName}\ \OPTG{\TRS{(}\ \OPT{\NT{PatternList}}\ \TRS{)}}\\
\NT{PatternList}   \defn \NT{Pattern}\ \MANYG{\TRS{,}\ \NT{Pattern}}
\end{abssyntax}

The case statement has the same pattern matching and binding semantics
as the case expression (see Section~\ref{sec:case-expr},
page~\pageref{sec:case-expr} for reference), except that shadowing local
variables is not allowed.


\section{Await Statement}
\emph{Await Statements} suspend the current task until the given
\emph{guard} is true~\cite{johnsen10fmco}. The task will not be
suspended if the guard is already initially true.  While the task is
suspended, other tasks within the same COG can be activated.  Await
statements are also called \emph{scheduling points}, because they, together with the Suspend statement, are
the only source positions where a task may become suspended and other
tasks of the same COG can be activated.


\begin{abssyntax}
\NT{AwaitStmt}  \defn \TR{await}\ \NT{Guard}\ \TRS{;}\\
\NT{Guard}      \defn \NT{ClaimGuard}
             \defc  \NT{PureExp}
             \defc  \NT{Guard}\ \TRS{\&}\ \NT{Guard}\\
\NT{ClaimGuard} \defn \NT{Variable}\ \TRS{?}
             \defc  \NT{FieldAccess}\ \TRS{?}
\end{abssyntax}

\begin{absexample}
Fut<Bool> f = x!m();
await f?;
await this.x == True;
await f? & this.y > 5;
\end{absexample}

\section{Suspend Statement}
A \emph{Suspend Statement} causes the current task to be suspended unconditionally.

\begin{abssyntax}
\NT{SuspendStmt} \defn \TR{suspend}\ \TRS{;}
\end{abssyntax}

\begin{absexample}
suspend;
\end{absexample}

\section{Skip Statement}
The \emph{Skip Statement} is a statement that does nothing.

\begin{abssyntax}
\NT{SkipStmt} \defn \TR{skip}\ \TRS{;}
\end{abssyntax}

\section{Assert Statement}\label{sec:abs:assert}
An \emph{Assert Statement} is a statement for asserting certain conditions.

\begin{abssyntax}
\NT{AssertStmt} \defn \TR{assert}\ \NT{PureExp}\ \TRS{;}
\end{abssyntax}

\begin{absexample}
assert x != null;
\end{absexample}

\section{Return Statement}
A \emph{Return Statement} defines the return value of a method.
A return statement can only appear as a last statement in a method body.

\begin{abssyntax}
\NT{ReturnStmt} \defn \TR{return}\ \NT{Exp}\ \TRS{;}
\end{abssyntax}

\begin{absexample}
return x;
\end{absexample}

\section{Expression Statement}
An \emph{Expression Statement} is a statement that only consists of a
single expression. Such statements are only executed for the effect of
the expression.  Expressions that can be used as statements are
documented in Chapter~\ref{cha:expr-with-side}.

\begin{abssyntax}
\NT{ExpStmt} \defn \NT{EffExp}\ \TRS{;}
\end{abssyntax}

\begin{absexample}
new C(x);
\end{absexample}

\section{Error Handling}
\label{sec:error-handling}

\emph{The information in this section is preliminary and subject to revision
  and redesign.  Open issues include: behavior of uncaught errors in
  synchronous method calls (delay killing the object / cog to allow the caller
  to handle the error vs killing the caller as well); the behavior of objects
  dying (are all processes terminated at the same time or eventually); the
  legality / behavior of return statements in \texttt{finally} blocks; etc.}

Errors are modeled via the \grsh{Exception} datatype.  New exceptions are
added to that type with an \grsh{exception} declaration (see
Section~\ref{sec:exceptions}).

The aim of modeling errors is to ensure a safe state as much as possible,
i.e., to make it difficult to inadvertently leave an object invariant violated
across a scheduling point.  Thus, any uncaught \emph{unexpected} errors
(receiving an error from a future, division by zero, \dots), as well as errors
raised via \texttt{die}, will result in the current object dying and all its
processes being terminated.  Any uncaught \emph{expected} errors (i.e., errors
explicitly raised via the \texttt{throw} statement) will result in the process
being terminated.

Error propagation in ABS is done via futures.  A method can either
terminate normally and return a value, or fail and return an error term.
When trying to read the value of a future that contains an error via
\texttt{get}, the error is raised on the caller's side but can be caught
and handled in a \texttt{catch} block.


\subsection{Throw Statement}
\label{sec:throw-statement}

The \texttt{throw} statement aborts normal execution of the process, raising
the exception specified via its argument.  When the throw occurs inside one or
more \texttt{try} blocks, the \texttt{catch} branches are tested against the
exception from inner to outer \texttt{try} block, interleaved with execution
of \texttt{finally} blocks.  It is the task of the catch- and finally-blocks
to re-establish the object invariant and perform corrective actions that
re-establish system-wide invariants.

If any \texttt{catch} branch matches the error, its statement(s) are executed
and execution continues as normal.  If the error reaches the outermost scope,
the process is killed.

\begin{abssyntax}
\NT{ThrowStmt} \defn \TR{throw}\ \NT{Exp}\ \TRS{;}
\end{abssyntax}


\subsection{Die Statement}
\label{sec:die-statement}

The \texttt{die} statement raises an error that, if uncaught, will kill an
object and all its processes.  All processes currently running on the object
will be terminated before they execute further, and callers will receive an
error.  All subsequent method calls to the killed object, be they synchronous
or asynchronous, will result in an error being returned to the caller.

\begin{abssyntax}
\NT{DieStmt} \defn \TR{die}\ \NT{Exp}\ \TRS{;}
\end{abssyntax}

Unexpected errors, i.e., errors not explicitly raised via the
\texttt{throw} statement, behave as if a \texttt{die} statement had been
executed in place of the statement resulting in the error.

\subsection{Try-Catch-Finally Statement}
\label{sec:try-catch-finally}

The \texttt{try-catch-finally} statement executes the enclosed statement
followed by the statement in the \texttt{finally} branch, if any.  (Both
these statements can be blocks.)  In case an error occurs during
execution of the statement(s) protected by \texttt{try}, the error is
matched against the patterns in the \texttt{catch} branches, in order.
The statements following the first matching catch branch are executed,
followed by the statements in the \texttt{finally} branch.  If none of the catch branches match, the statements in the \texttt{finally} branch are executed and the error is re-thrown.

Code in the \texttt{finally} branch has the same restrictions as init blocks:
no suspension, blocking or \texttt{return} statement is allowed.

\begin{abssyntax}
\NT{TryCatchFinallyStmt} \defn \TR{try}\ \NT{Stmt}\
                               \TR{catch} \TRS{\{}
                                 \MANY{\NT{CaseStmtBranch}}\
                               \TRS{\}}
                               \OPTG{\TR{finally} \NT{Stmt}}\\
\NT{CaseStmtBranch}    \defn \NT{Pattern}\ \TRS{=>} \NT{Stmt} \\
\NT{Pattern}       \defn \NT{Identifier}
                \defc  \NT{Literal}
                \defc  \NT{ConstrPattern}
                \defc  \TRS{\_}\\
\NT{ConstrPattern} \defn \NT{TypeName}\ \OPTG{\TRS{(}\ \OPT{\NT{PatternList}}\ \TRS{)}}\\
\NT{PatternList}   \defn \NT{Pattern}\ \MANYG{\TRS{,}\ \NT{Pattern}}
\end{abssyntax}

The following example shows the usage of \texttt{throw} and
\texttt{try-catch-finally}.

\begin{absexample}
exception MyException;
exception AnotherException;

interface I {
    Int doit();
}

class C implements I {
    Int doit() {
        throw MyException;
        return 5;
    }
}

{
    I i = new C();
    String step1 = "unexecuted";
    String step2 = "uncaught";
    String step3 = "unfinalized";
    Int x = -1;
    try {
        step1 = "executed";
        x = await i!doit();
    } catch {
        AnotherException => step2 = "caught the wrong thing";
        MyException => step2 = "caught";
        _ => step2 = "did not catch the right thing";
    } finally {
        step3 = "finalized";
    }
}
\end{absexample}

% Local Variables:
% TeX-master: "absrefmanual"
% End:
