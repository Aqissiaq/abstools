\chapter{Pure Expressions}
\emph{Pure Expressions} are side effect-free expressions. This means
that these expressions cannot modify state variables.

\begin{abssyntax}
\NT{PureExp}     \defn \NT{Variable}
              \defc  \NT{FieldAccess}
              \defc  \NT{ThisExp}
              \defc  \NT{NullLiteral}
              \defc  \NT{LetExp}
              \defc  \NT{DataConstrExp}
              \defc  \NT{FnAppExp}
              \defc  \NT{FnAppListExp}
              \defc  \NT{IfExp}
              \defc  \NT{CaseExp}
              \defc  \NT{OperatorExp}
              \defc  \TRS{(} \NT{PureExp} \TRS{)}\\
\NT{Variable}    \defn \NT{Identifier}\\
\NT{FieldAccess} \defn \NT{ThisExp}\ \TRS{.}\ \NT{Identifier}\\
\NT{ThisExp}     \defn \TR{this}\\
\NT{PureExpList} \defn \NT{PureExp}\ \MANYG{\TRS{,}\ \NT{PureExp}}
\end{abssyntax}


\section{Let Expressions}
\emph{Let Expressions} bind variable names to pure expressions.

\begin{abssyntax}
\NT{LetExp} \defn \TR{let}\ \TRS{(} \NT{Param} \TRS{)}\ \TRS{=}\ \NT{PureExp}\ \TR{in}\ \NT{PureExp}
\end{abssyntax}

\begin{absexample}
let (Bool x) = True in !x  
\end{absexample}

\section{Data Type Constructor Expressions}
\emph{Data Type Constructor Expressions} are expressions that create data type values by using data type constructors.
Note that for data type constructors that have no parameters, the parentheses are optional.

\begin{abssyntax}
\NT{DataConstrExp} \defn \NT{TypeName}
                   \defc \NT{TypeName}\ \TRS{(} \OPT{\NT{PureExpList}} \TRS{)}
\end{abssyntax}

\begin{absexample}
True
Cons(True, Nil)  
ABS.StdLib.Nil
\end{absexample}

\section{Function Applications}
\emph{Function Applications} apply functions to arguments.

\begin{abssyntax}
\NT{FnAppExp} \defn \NT{Name}\ \TRS{(} \OPT{\NT{PureExpList}} \TRS{)}   
\end{abssyntax}

\begin{absexample}
tail(Cons(True, Nil))
ABS.StdLib.head(list)
\end{absexample}

\section{If-Then-Else Expression}
ABS has a standard if-then-else expression.

\begin{abssyntax}
\NT{IfExp}  \defn \TR{if}\ \NT{PureExp}\ \TR{then} \NT{PureExp}\ \TR{else} \NT{PureExp}
\end{abssyntax}


\begin{absexample}
if 5 == 4 then True else False
\end{absexample}

\section{Case Expressions / Pattern Matching}
\label{sec:case-expr}
ABS supports pattern matching by the \emph{Case Expression}.  It takes
an expression as first argument, which is matched against a series of
patterns.  The value of the case expression itself is the value of the
expression on the right-hand side of the first matching pattern.
It is an error if no pattern matches the expression.


\begin{abssyntax}
\NT{CaseExp}       \defn \TR{case}\ \NT{PureExp}\ \TRS{\{} \MANY{\NT{CaseExpBranch}}\ \TRS{\}}\\
\NT{CaseExpBranch}    \defn \NT{Pattern}\ \TRS{=>}\ \NT{PureExp}\ \TRS{;}\\
\NT{Pattern}       \defn \NT{Identifier}
                \defc  \NT{Literal}
                \defc  \NT{ConstrPattern}
                \defc  \TRS{\_}\\
\NT{ConstrPattern} \defn \NT{TypeName}\ \OPTG{\TRS{(}\ \OPT{\NT{PatternList}}\ \TRS{)}}\\
\NT{PatternList}   \defn \NT{Pattern}\ \MANYG{\TRS{,}\ \NT{Pattern}}
\end{abssyntax}

% \begin{abscode}
% case <expr> {
%   <pattern> => <expr> ;
%   <pattern> => <expr> ;
%   ...
% }  
% \end{abscode}

\subsection{Patterns}
There are five different kinds of patterns available in ABS:
\begin{itemize}
\item Pattern Variables (e.g., \absinline{x}, where \absinline{x} is not bound yet)
\item Bound Variables (e.g., \absinline{x}, where \absinline{x} is bound)
\item Literal Patterns (e.g., \absinline{5})
\item Data Constructor Patterns (e.g.,~\absinline{Cons(Nil,x)})
\item Underscore Pattern (\absinline{_})
\end{itemize}

\subsubsection{Pattern Variables}
Pattern variables are simply unbound variables.
Like the underscore pattern, these variables match every value, but, in addition, bind the variable to the matched value. The bound variable can then be used in the right-hand-side expression of the corresponding branch.
Typically, pattern variables are used inside of data constructor patterns to extract values from data constructors.
For example:
\begin{abscode}
def A fromJust<A>(Maybe<A> a) = 
  case a { 
    Just(x) => x; 
  };
\end{abscode}

\subsubsection{Bound Variables}
If a bound variable is used as a pattern, the pattern matches if the value of the case expression is equal to the value of the bound variable.

\begin{abscode}
def Bool contains<A>(List<A> list, A value) =
  case list {
    Nil => False;
    Cons(value, _) => True;
    Cons(_, rest) => contains(rest, value);
  };
\end{abscode}

\subsubsection{Literal Patterns}
Literals can be used as patterns. This is similar to bound variables, because the pattern matches if the value of the case expression is equal to the literal value.

\begin{abscode}
def Bool isEmpty(String s) =
  case b {
    "" => True;
    _  => False;
  };
\end{abscode}

\subsubsection{Data Constructor Patterns}
A data constructor pattern is like a standard data constructor expression, but where certain sub expressions can be patterns again.

\begin{abscode}
def Bool negate(Bool b) =
  case b {
    True => False;
    False => True;
  };
\end{abscode}

\begin{abscode}
def List<A> remainder(List<A> list) =
  case b {
    Cons(_, rest) => rest;
  };
\end{abscode}

\subsubsection{Underscore Pattern}
The underscore pattern (\absinline{_}) simply matches every value, but
in contrast to pattern variables the matched value is not retained in
the right-hand-side expression of the branch.  It is generally used as
the last pattern in a case expression to define a default case.  For
example:
\begin{abscode}
def Bool isNil<A>(List<A> list) =
  case list {
    Nil => True;
    _ => False;
  };
\end{abscode}


% \subsubsection{Error Cases}
% It is currently backend dependent what happens if there is a case expression which does not match any pattern.
% The Java backend throws an \verb_UnmatchedCaseException_ exception in this case.

\subsection{Type Checking}
A case expression is type-correct if and only if all its expressions and all its branches are type-correct and the right-hand side of all branches have a common super type. This common super type is also the type of the overall case expression. 

A branch (a pattern and its expression) is type-correct if its pattern and its right-hand side expression are type-correct.
A pattern is type-correct if it can match the corresponding case expression.

\section{Operator Expressions}
ABS has a number of predefined operators which can be used to form \emph{Operator Expressions}.

\begin{abssyntax}
\NT{OperatorExp} \defn \NT{UnaryExp}
              ~|~ \NT{BinaryExp}\\
\NT{UnaryExp}    \defn \NT{UnaryOp}\ \NT{PureExp}\\
\NT{UnaryOp}     \defn \verb*_!_ ~|~ \TRS{-}\\
\NT{BinaryExp}   \defn \NT{PureExp}\ \NT{BinaryOp}\ \NT{PureExp}\\
\NT{BinaryOp}    \defn \TRS{||} ~|~\TRS{\&\&} ~|~ \TRS{==} ~|~ \TRS{!=} ~|~ \TRS{<} ~|~ \TRS{<=} ~|~ \TRS{>} ~|~ \TRS{>=} ~|~ \TRS{+} ~|~ \TRS{-} ~|~ \TRS{*} ~|~ \TRS{/} ~|~ \TRS{\%}
\end{abssyntax}

\begin{table}[ht]
\centering
 \renewcommand{\arraystretch}{1.5} 
 \begin{tabular}[h]{| l | l | l | l | l |}
   \hline
   Expression & Meaning & Associativity & Argument types & Result type \\
   \hline
\mcode{e1 || e2}  & logical or & left & Bool, Bool & Bool \\ \hline 
\mcode{e1 \&\& e2}  & logical and & left & Bool, Bool & Bool \\ \hline
\mcode{e1 == e2}& equality & left & compatible & Bool \\ 
\mcode{e1 != e2}& inequality & left & compatible & Bool \\ \hline 
\mcode{e1 < e2}& less than & left & compatible & Bool \\
\mcode{e1 <= e2}& less than or equal to  & left & compatible & Bool \\
\mcode{e1 > e2}& greater than & left & compatible & Bool \\
\mcode{e1 >= e2}& greater than or equal to  & left & compatible & Bool \\ \hline
\mcode{e1 + e2}& concatenation & left & String, String & String \\
\mcode{e1 + e2}& addition & left & number, number & number \\
\mcode{e1 - e2}& subtraction & left & number, number & number \\\hline
\mcode{e1 * e2}& multiplication & left & number, number & number \\
\mcode{e1 / e2}& division & left & number, number & Rat \\
\mcode{e1 \% e2}& modulo & left & number, number & Int \\ \hline 
\verb_! e_  & logical negation  & right & Bool  & Bool \\ 
\verb_- e_   & integer negation  & right & number  & number \\ \hline 
 \end{tabular}
  \caption{\label{fig:operatorExpressions} Operator expressions, grouped according to precedence from low to high. ``number'' is either Int or Rat.}
\end{table} 
Table~\ref{fig:operatorExpressions} describes the meaning as well as the associativity and the precedence of the different operators.
They are grouped according to precedence, as indicated by horizontal
rules, from low precedence to high precedence.

\subsection{Semantics of comparison operators}

Basic datatypes (numbers, strings) are compared by value.

Pre-defined and user-defined algebraic datatypes are compared by value.
For \absinline{<} etc., an ordering is established via recursive
lexicographical comparison on constructor name and arguments (i.e.,
\absinline{Cons(...) < Nil} and \absinline{Cons(1, Nil) < Cons(2,
  Nil)}).

Objects and futures are compared by identity.  The ordering comparison
operators (\absinline{<} etc.) define an arbitrary but stable ordering,
i.e., the order does not change when an object or future changes
state.\footnote{This ordering might not be stable between two
  invocations of a program.  If ABS ever develops object serialization,
  care must be taken to uphold any datatype invariants across program
  invocations, e.g., when reading back an ordered list of objects.}



%\lstDeleteShortInline|

%%% Local Variables:
%%% mode: latex
%%% TeX-master: "absrefmanual"
%%% End:



