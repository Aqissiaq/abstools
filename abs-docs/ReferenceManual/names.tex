\chapter{Names and Types}
\section{Names}
A \emph{name} in ABS can either be a simple identifier as described above, or can be qualified with a type name, which represents a module. 

\begin{abssyntax}
\NT{TypeName}     \defn \NT{TypeId}\ \MANYG{\TR{.}\ \NT{TypeId}}\\
\NT{Name}         \defn \NT{Identifier} ~|~ \NT{TypeName}\ \TR{.}\ \NT{Name}
\end{abssyntax}

Examples for syntactically valid names are: \absinline{head}, \absinline{x}, \absinline{ABS.StdLib.tail}. Examples for type names are: \absinline{Unit}, \absinline{X}, \absinline{ABS.StdLib.Map}.

\section{Types}
\emph{Types} in ABS are either plain type names or can have type arguments.

\begin{abssyntax}
\NT{Type}     \defn \NT{TypeName}\ \OPT{\NT{TypeArgs}}\\
\NT{TypeArgs} \defn \TRS{<}\ \NT{TypeList}\ \TRS{>}\\
\NT{TypeList} \defn \NT{Type}\ \MANYG{\TR{,}\ \NT{Type}}
\end{abssyntax}

Where \absinline{TypeName} can refer to a data type, an interface, a type synonym, and a
type parameter. Note that classes cannot be used as types in ABS.
In addition, only parametric data types can have type arguments.
Examples for syntactically valid types are: \absinline{Bool}, \absinline{ABS.StdLib.Int}, 
\absinline{List<Bool>},
\absinline{ABS.StdLib.Map<Int,Bool>}.


\section{Type Synonyms} \label{sec:typesynonyms}
\emph{Type Synonyms} define synonyms for otherwise defined types. 
Type synonyms start with an uppercase letter.

\begin{abssyntax}
\NT{TypeSynDecl} \defn \TRS{type} \NT{TypeId}\ \TRS{=}\ \NT{TypeName}\ \TRS{;}  
\end{abssyntax}

%\begin{absexample}
%type Filename = String; 
%type IntList = List<Int>;
%\end{absexample}
\begin{absexample}
type Filename = String
type Filenames = Set<Filename>
type Servername = String
type Packet = String
type File =  List<Packet>
type Catalog = List<Pair<Servername,Filenames>>
\end{absexample}


% Local Variables:
% TeX-master: "absrefmanual"
% End:
