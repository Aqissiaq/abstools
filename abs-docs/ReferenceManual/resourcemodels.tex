\chapter{Deployment Models}\label{ch:resourcemodels}


\section{Introduction}

  This section describes language features related to deployment
  components.  Most of these features are implemented on the Maude
  backend and ignored on the Java backend.

  ABS models can be augmented with resource information and the effects
  of resources during execution can be simulated on the Maude backend.
  Resources are contained in \emph{deployment components}, which provide a
  context in which concurrent object groups and their objects execute.
  Deployment components can be used to model servers, virtual machines,
  nodes in a network, and basically any entity that constrains execution
  of code running in its context.

  At the moment we only model CPU capacity.  A deployment component (DC)
  can contain a certain ``CPU capacity'', which provides an abstract model
  of speed relative to other DCs; i.e., a cpu capacity of 50 is ``twice
  as fast'' as 25.

  Executing statements costs a certain number of resources, which are
  refilled in each time interval -- since CPU speed is related to
  performance and hence time, simulations involving performance
  characteristics must be done using the {{{./timedmodeling.html}timed
  interpreter}}.

\section{Language Elements}

  All identifiers described in the following are contained in the module
  \absinline{ABS.DC}.

\subsection{Resource Types}

Deployment components carry a specific capacity of zero or more \emph{Resource
  types}.  Currently, we model CPU and bandwidth as resources.  CPU resources
are modeled as a possibly infinite number of CPU ``cycles'', bandwidth as a possibly infinite amount of transfer units per time slice.

\begin{absexample} 
data InfRat = InfRat | Fin(Rat value);
data Resourcetype = Speed | Bandwidth ;
\end{absexample} 

\subsection{Deployment Components}

A deployment component (``DC'') is modeled by an ABS class and interface
\absinline{DeploymentComponent} which is contained in the standard library in
the module \absinline{ABS.DC}.  DCs are initialized with resource data as a
map from resource type to amount.  Unspecified resource types are assumed to
be infinite.  This resource information is used by the Maude backend to
simulate cpu and bandwidth usage and performance behavior of models.

\begin{absexample} 
interface DeploymentComponent {
    [Atomic] InfRat available(Resourcetype rtype);
    [Atomic] Rat load(Resourcetype rtype, Int periods);
    [Atomic] InfRat total(Resourcetype rtype);
    Unit transfer(DeploymentComponent target, Rat amount, Resourcetype rtype);
    Unit decrementResources(Rat amount, Resourcetype rtype);
    Unit incrementResources(Rat amount, Resourcetype rtype);
}

class DeploymentComponent (String description,
                           Map<Resourcetype, Rat> initconfig)
implements DeploymentComponent {
    ...
}
\end{absexample} 

\begin{itemize}
\item The method \absinline{available(r)} returns the number of currently
  available resources for the specified resource type $r$.  (The value will
  change during execution, hence code using this method should be aware of the
  obvious race condition.)
\item The call \absinline{load(r, n)} returns the \emph{average load} for the
  last $n$ time periods as an integer between 0 and 100 for the specified
  resource type $r$.  For infinite resources, \absinline{load} returns zero.
\item The method \absinline{total(r)} returns the number of resources currently
  available per time unit for the resource type $r$.
\item The method \absinline{transfer(target, amount, r)} transfers the
  specified number of resources of type $r$ from the current DC to
  \absinline{target}.  If the current or the target DC have an infinite number
  of resources, their respective values are not changed.
\item The methods \absinline{decrementResources} and
  \absinline{incrementResources} are used to implement \absinline{transfer},
  but can also used on their own to increment or decrement the amount of
  resources of the current deployment component.
\end{itemize}

\subsection{Using Deployment Components}

   An optional annotation \absinline{[DC: x]} attached to a \absinline{new}
   statement expresses that the new object's cog will run in the context
   of deployment component \absinline{x}.  By default, a new cog runs in the
   same context as the process that generates it.  The main block runs
   in a DC with no resource limits.

\begin{absexample} 
interface Server { ... }

class ServerImp implements Server { ... }

{
  DeploymentComponent machine = new DeploymentComponent("remote",
      map[Pair(Speed, 20)]);
  [DC: machine] Server x = new ServerImp();  // running in dc `machine'
  Server y = new ServerImp();                // running in current DC
}
\end{absexample} 

\subsection{Accessing the Current Deployment Component}

\begin{absexample} 
def DeploymentComponent thisDC() = builtin;
\end{absexample} 

   The \absinline{thisDC()} function returns the current deployment
   component, i.e., the one that was given as annotation at the \absinline{new}
   object instantiation.  \absinline{thisDC} returns \absinline{null} if no
   deployment component was specified.  (Due to implementation issues,
   models containing \absinline{thisDC()} cannot be currently compiled on the
   Java platform.)

\subsection{Expressing statement costs}

The annotation \absinline{[Cost: x] s} expresses that executing the statement
\absinline{s} will consume $x$ Speed resources on the current deployment
component.  If the current DC does not contain enough resources, the process
will \emph{block}, consuming Speed resources each time unit, until the necessary
amount of resources have been consumed.

The annotation \absinline{[DataSize: x] o!m();} expresses that executing the
asynchronous method call \absinline{o!m()} will consume $x$ bandwidth
resources, both on the current and the target deployment components.
Execution will block on the current DC until the specified amount of resources
have been consumed.  The method call will be delivered when $x$ bandwidth
resources have been consumed at the target DC, but never before the current DC
has finished executing the asynchronous method call statement.

When a statement has both \absinline{Cost} and \absinline{Size} annotations,
Speed resources will be consumed first, followed by bandwidth resources.

\subsection{Default statement costs}

By default, statement execution costs nothing (i.e., the machine is infinitely
fast), which preserves the normal semantics of timed ABS where only explicit
\absinline{duration} or \absinline{await duration} statements advance the
clock.  By compiling ABS models with the command-line parameter
\absinline{-defaultcost=n}, all statements without an explicit
\absinline{[Cost: x]} annotation will take \absinline{n} resources to execute.
This approximates ``normal'' CPU behavior a bit more closely, but still
disregards the cost of evaluating expressions (assignment of a complex
expression will take more time on a real machine than assignment of a
constant).

\subsection{Changing the deployment component}

A statement \absinline{movecogto dc;} transfers the current cog and all its
objects to the deployment component \absinline{dc}.  On backends without
resource modeling features (Java, Scala) this translates to a
\absinline{skip}.

\section{Future Work}

The \absinline{thisDC()} function is currently undefined in the Java
backend and will lead to a compile-time error.

% Local Variables:
% TeX-master: "absrefmanual"
% End:
